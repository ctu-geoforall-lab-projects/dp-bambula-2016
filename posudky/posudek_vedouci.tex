\documentclass[czech,11pt,a4paper]{article}
\usepackage[utf8]{inputenc}
\usepackage{a4wide}
\usepackage[pdftex,breaklinks=true,colorlinks=true,urlcolor=blue,
  pagecolor=black,linkcolor=black]{hyperref}
\usepackage[czech]{babel}

\pagestyle{empty}

\begin{document}

\begin{center}
  {\Large --- Posudek vedoucího diplomové práce ---}
\end{center}

\vspace{.2cm}

\noindent \begin{tabular}{rp{.9\textwidth}}
  {\bf Název:} & Rozšíření nástroje pro práci s katastrálními daty v programu QGIS \\
  {\bf Student:} & Bc. Štěpán Bambula \\
  {\bf Vedoucí:} & Ing. Martin Landa, Ph.D. \\
  {\bf Fakulta:} & Fakulta stavební ČVUT v Praze \\ 
  {\bf Katedra:} & Katedra geomatiky \\
  {\bf Oponent:} & Ing. Arnošt Müller \\
  {\bf Pracoviště oponenta:} & Státní pozemkový úřad v Praze \\
\end{tabular}

\vspace{1cm}

Práce navazuje na dlouholetý projekt laboratoře Open Source Geospatial
Research and Education Laboratory (OSGeoREL) na katedře geomatiky,
fakulty stavební ČVUT v Praze. Jeho cílem je vývoj a údržba
specializovaného nástroje pro práci s katastrálními daty poskytovanými
ve výměnném formátu katastru (VFK) na open source GIS platformě
označované jako QGIS. Prototyp tohoto nástroje vznikl v roce 2012 jako
semestrální projekt předmětu Projekt informatika 2. Tento prototyp byl
dále rozvíjen v~navazujícím semestrálním projektu v~roce 2015, jehož
spoluautorem byl předkladatel práce. Hlavním cílem projektu bylo usnadnění
distribuce nástroje k cílovým uživatelům. Vzhledem k tomu, že QGIS
podporuje snadnou distribuci pouze nástrojů (tzv. zásuvných modulů)
napsaných v programovacím jazyku Python, padla volba na tuto
platformu. V rámci semestrálního projektu byla portována pouze část
nástroje. Student tuto část nejprve musel dokončit. Na základě toho
vzniklo zadání předkládané práce. \newline

V rámci práce projekt, který vzhledem k tomu, že se přes
nejrůznější technické problémy těžko dostával ke svým uživatelům a
realisticky řečeno skomíral, byl opět oživen. Díky tomu je nástroj snadno
dostupný pro běžného uživatele a stal se z dlouhodobého pohledu pro
jeho autory udržitelným. Tento výsledek je zcela zásadní pro další
vývoj projektu, který je současně jakousi výkladní skříní laboratoře
OSGeoREL v Praze a úspěšnou prezentací výsledků práce studentů na
studijním oboru Geomatika. \newline

Z hlediska hodnocení předkládané práce jsou podstatná rozšíření, která
student do projektu přinesl. Jedná se o načítání více souborů VFK do
jedné pracovní databáze a především o podporu zpracování změnových
souborů. Nejprve byly nutné úpravy provedeny na straně ovladače GDAL,
který umožňuje data ve formátu VFK zásuvnému modulu QGIS načíst. Tato
část byla zajištěna vedoucím práce. Úpravy a implementace nových
funkcionalit na straně zásuvného modulu byly provedeny
studentem. Výsledkem je nová verze nástroje označovaná jako 2.1,
která umožňuje zpracovat více souborů ve formátu VFK najednou. Což je
vzhledem k tomu, že jsou katastrální data daného zájmového území běžně
poskytována ve více souborech, zcela zásadní pro jeho další uplatnění
mezi potencionálními uživateli. Druhé podstatné rozšíření se týká
zpracování změnových vět. Logika zpracování změnových dat byla
implementována studentem v rámci knihovny v jazyce
Python, která je použita jak v zásuvném modulu QGIS, tak i v nástroji
pro příkazovou řádku. Oficiální zveřejnění nové verze nástroje se
předpokládá na podzim roku 2016, kdy vyjde i verze knihovny GDAL 2.2,
na které jsou nové functionality závislé. \newpage

Student projevil dlouhodobý zájem na projektu pracovat, vnesl do něj
svoji iniciativu jejímž výsledkem je stabilní verze nástroje 2.0 a
vývojová verze 2.1, která přináší nové, pro praxi důležité
funkcionality. Ze stagnujícího projektu se díky tomu stal živý
projekt prezentující práci studentů na našem oboru a zároveň platformu
pro jeho další vývoj. Tento přínos považuji pro moje
hodnocení za zásadní. \newline

Na základě výše uvedeného hodnotím předloženou diplomovou práci
klasifikačním stupněm

\begin{center}
  {\bf --- A (výborně) --- }
\end{center}

\vskip 2cm

\begin{tabular}{lp{.2\textwidth}r}
& & \ldots\ldots\ldots\ldots\ldots\ldots\ldots \\
V~Solanech dne 17. června 2016 & & Ing. Martin Landa, Ph.D. \\
& & Fakulta stavební, ČVUT v Praze \\
\end{tabular}

\end{document}
