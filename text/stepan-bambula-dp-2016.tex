%----------------------------------------------------------------------------------------
%	PACKAGES AND OTHER DOCUMENT CONFIGURATIONS
%----------------------------------------------------------------------------------------

%\documentclass[12pt]{article}
\documentclass[a4paper,12pt,oneside]{book}

\usepackage[utf8]{inputenc}	% kódování textu
\usepackage[czech]{babel}		% zavedení češtiny
\usepackage[IL2]{fontenc}
\usepackage[pdftex]{hyperref}	% veškeré klikací odkazy
\usepackage{a4wide}
\usepackage{indentfirst}	% odsazení prvního řádku odstavce
\usepackage{amsmath,amsfonts,amssymb}	% matematika
\usepackage{graphicx}	% grafika
\usepackage{multirow}	% slouceni radek v tabulce
\usepackage{multicol}	% slouceni sloupcu v tabulce
\usepackage{longtable}	% rozdeleni tabulky pres vice stran
\usepackage{enumerate}	% seznamy
\usepackage{float}
\usepackage{booktabs}	% professional tables
\usepackage{lscape}		% stranka na sirku
\usepackage{fancyhdr}
\usepackage{url}
\usepackage{array}
\usepackage{wasysym}
\usepackage{listings} 
\usepackage{color}
\usepackage{caption}
\usepackage{subcaption}

\definecolor{light-gray}{gray}{0.97}

\lstset{
numbers=left,
numberstyle=\tiny,
basicstyle=\ttfamily,
columns=flexible,
breaklines=true,
backgroundcolor=\color{light-gray},
xleftmargin=\parindent,
}

\renewcommand\lstlistingname{Kód}

\usepackage[%
%top=40mm,
%bottom=35mm,
%left=40mm,
%right=30mm
top=40mm,
bottom=35mm,
left=35mm,
right=25mm
]{geometry}

% nastavení příkazu URL podle normy ISO
%\DeclareUrlCommand\url{\def\UrlLeft{< }\def\UrlRight{ >} \urlstyle{tt}}	% správné zobrazení www

\setlength{\parskip}{1ex}


\begin{document}
% ===============================================================
% NASTAVENI ZAHLAVI A ZAPATI	
% ===============================================================
% uvodni stranka zahlavi ani zapati mit nebude
\thispagestyle{empty}
 % vlevo text a název aktuální sekce
\lhead{\includegraphics[scale=0.2]{images/cvut_logo.png}\hspace{10pt}ČVUT v~Praze}
 % vpravo název kapitoly
\rhead{{\rightmark}}

 % nastavime pouziti naseho stylu
\pagestyle{fancy}

\renewcommand{\chaptermark}[1]{\markright{#1}{}}
%\renewcommand{\sectionmark}[1]{\markright{#1}{}}

\fancypagestyle{plain}{
  \fancyhead{} % na prázdných stránkách nechci záhlaví
  \renewcommand{\headrulewidth}{0pt} % ani linku
}

% =========================
% TITULNÍ STRANA 11111111111
% =========================

\begin{titlepage}
 
\newcommand{\HRule}{\rule{\linewidth}{0.5mm}} % Defines a new command for the horizontal lines, change thickness here

\center % Center everything on the page

%----------------------------------------------------------------------------------------
%	HEADING SECTIONS
%----------------------------------------------------------------------------------------

\textsc{\LARGE České vysoké učení technické v~Praze}\\[0.5cm] % Name of your university/college
\textsc{\Large Fakulta stavební} % Major heading such as course name


%----------------------------------------------------------------------------------------
%	TITLE SECTION
%----------------------------------------------------------------------------------------

\vfill

\textsc{\LARGE Diplomová práce}

%----------------------------------------------------------------------------------------
%	AUTHOR SECTION
%----------------------------------------------------------------------------------------
\vfill
\begin{minipage}{0.4\textwidth}
\begin{flushleft} 
\large 2016	% datum
\end{flushleft}
\end{minipage}
~
\begin{minipage}{0.4\textwidth}
\begin{flushright} 
\large Bc. Štěpán \textsc{Bambula} \\ % Your name
\end{flushright}
\end{minipage}\\[1cm]

\end{titlepage}
% -------------------------------------------------------------------------------------------------------------


% =========================
% TITULNÍ STRANA 2222222222
% =========================

\begin{titlepage}
\center % Center everything on the page
 
%----------------------------------------------------------------------------------------
%	HEADING SECTIONS
%----------------------------------------------------------------------------------------

\textsc{\LARGE České vysoké učení technické v~Praze}\\[0.5cm] % Name of your university/college
\textsc{\Large Fakulta stavební}\\[0.5cm] % Major heading such as course name
\textsc{\large Studijní program: Geodézie a~kartografie}\\[0.5cm] % Minor heading such as course title
\textsc{\large Studijní obor: Geomatika}\\[0.5cm] % Minor heading such as course title

%----------------------------------------------------------------------------------------
%	LOGO SECTION
%----------------------------------------------------------------------------------------

\vspace{50pt}

\includegraphics[scale=1.2]{images/cvut_logo.png}  % Include a department/university logo 

%----------------------------------------------------------------------------------------
%	TITLE SECTION
%----------------------------------------------------------------------------------------

\vspace{40pt}

\textsc{\Large Diplomová práce}\\
\vfill
\textsc{\LARGE  Rozšíření nástroje pro práci s~katastrálními daty v~programu QGIS}\\[0.5cm] % Title of your document

\textsc{\Large QGIS VFK Plugin Improvements}
 
 
 %----------------------------------------------------------------------------------------
%	VEDOUCÍ PRÁCE
%----------------------------------------------------------------------------------------
 \vfill
\large Vedoucí práce: Ing. Martin \textsc{Landa}, Ph.D.

Katedra geomatiky
 
%----------------------------------------------------------------------------------------
%	AUTHOR SECTION
%----------------------------------------------------------------------------------------
\vfill
\begin{minipage}{0.4\textwidth}
\begin{flushleft} 
\large 2016	% datum
\end{flushleft}
\end{minipage}
~
\begin{minipage}{0.4\textwidth}
\begin{flushright} 
\large Bc. Štěpán \textsc{Bambula} \\ % Your name
\end{flushright}
\end{minipage}\\[1cm]
 
\end{titlepage}

%========================= ZADÁNÍ PRÁCE =================================
\clearpage
\pagestyle{empty}

\vspace*{\fill}
\begin{center}
\textsc{\Large Zde vložit zadání práce!!!}
\end{center}

\vspace*{\fill}

%========================= ABSTRAKT =====================================
\clearpage

\hfill

\noindent
\textsc{\Large Abstrakt}

\vspace{12pt}

Cílem diplomové práce je rozšířit projekt laboratoře OSGeoREL ČVUT v~Praze zaměřený na práci s~katastrálními daty poskytovanými ve výměnném formátu VFK v~prostředí open source nástroje QGIS. Práce navazuje na již existující nástroj implementovaný jako tzv. zásuvný modul a~rozšiřuje ho o~novou funkctionalitu a~to především zpracování a~vizualizaci datových vět změnových souborů VFK. Druhotným cílem je usnadnění distribuce zásuvného modulu v~prostředí QGIS s~důrazem na jeho přenositelnost.

\vspace{32pt}

\noindent
\textsc{\Large Klíčová slova}

\vspace{12pt}

VFK, QGIS, ČUZK, Python, C++, PyQt, GDAL, zásuvný modul


\vfill

\noindent
\textsc{\Large Abstract}

\vspace{12pt}


\vspace{32pt}

\noindent
\textsc{\Large Keywords}

\vspace{12pt}
VFK, QGIS, CUZK, Python, C++, PyQt, GDAL, plugin


\vfill

%========================= PROHLÁŠENÍ ==================================
\clearpage
\vspace*{\fill}

\noindent
\textsc{\Large Prohlášení}

\vspace{12pt}
Prohlašuji, že jsem diplomovou práci na téma \uv{Rozšíření nástroje pro práci s~katastrálními daty v~programu QGIS} vypracoval samostatně. Všechny podklady, ze kterých jsem čerpal, jsou uvedeny v~seznamu použité literatury.

\vspace{24pt}
\noindent
\begin{minipage}{0.4\textwidth}
\begin{flushleft}
\center 
V~Praze dne \dots \dots \dots \\
\end{flushleft}
\end{minipage}
~
\begin{minipage}{0.8\textwidth}
\begin{flushright} 
\vspace{20pt}
\center
\dots \dots \dots \dots \dots \dots \dots \dots \\
Štěpán Bambula
\end{flushright}
\end{minipage}\\[2cm]


%========================= PODĚKOVÁNÍ ==================================
\clearpage
\vspace*{\fill}

\noindent
\textsc{\Large Poděkování}

\vspace{12pt}

\vspace{2cm}

%======================POUZITE ZKRATKY===============================
\clearpage
%\rhead{SEZNAM POUŽITÝCH ZKRATEK}		 % vpravo název kapitoly
\chapter*{Seznam použitých zkratek}
\thispagestyle{empty}

\begin{description}
\item[VFK] Výměnný formát katastru nemovitostí
\item[ČUZK] Český úřad zeměměřický a~katastrální
\item[GDAL] Geospatial Data Abstraction Library
\item[GIS] Geografický informační systém
\item[OSGeo] Open Source Geospatial Foundation
\item[ISKN] Informační systém katastru nemovitostí
\item[SGI] Soubor geodetických informací
\item[SPI] Soubor popisných informací
\end{description}


%=========================OBSAH=======================================
\clearpage
\rhead{{\rightmark}}		% vpravo název kapitoly
\tableofcontents
\thispagestyle{empty}

%============================ÚVOD====================================
\clearpage
\pagestyle{fancy}		% nastaví styl stránky a číslování
\setcounter{page}{1}   	% nastaví čítač stránek znovu od jedné
\pagenumbering{arabic} % číslování arabskými
\rhead{Úvod}		 % vpravo název kapitoly
\chapter*{Úvod}
\addcontentsline{toc}{chapter}{Úvod}



\clearpage
\rhead{{\rightmark}}
\chapter{Rešerše nástrojů pro práci s~VFK}

V~současné době existuje několik nástrojů pro práci s~katastrálními daty. Ať už se jedná o~moduly, které jsou distribuovány zdarma, nebo za úplatu,  jako nadstavba pro komerční nástroje, případně o~moduly dostupné pro open-source nástroje. S~trochou nadsázky se dá říci, že každý větší software pro práci s~prostorovými daty má svůj zásuvný modul (nadstavbu) pro práci s~daty \textit{VFK}.

V~této kapitole bych se chtěl věnovat vybraným nástrojům pracujícím s~těmito daty. Moduly, které jsou šířeny zdarma a~bylo je možné vyzkoušet, byly otestovány přímo. Popis komerčních modulů vychází především z~dostupně dokumentace na příslušných oficiálních webových stránkách.

\section{ISKN Studio pro ArcGIS}
\label{l_iskn_studio}
% TODO: https://www.arcdata.cz/produkty/ceska-specifika/iskn-pro-arcgis-for-desktop
Aplikace ISKN Studio slouží pro převod dat VFK do do geodatabáze, se kterou je schopen pracovat program ArcGIS. Struktura geodatabáze je navržena dle připravené šablony. Tato šablona je vždy vázána k~dané verzi formátu VFK a~může být stažena přímo ze stránek společnosti Arcdata Praha, která je zárověn zvůrcem této aplikace. Data načtená touto aplikací mohou být exportována také ve formátu \textit{.xml}. Díky této aplikaci může být v~geodatabázi sestavena geometrie vektorových prvků. \cite{iskn_studio}

\begin{figure}[htb]
\centering
\includegraphics[width=\textwidth]{images/ISKNStudio-aplikace.png}
\caption[Aplikace ISKN Studio]{Aplikace ISKN Studio (zdroj: vlastní)}
\end{figure}

Strutura vytvořené geodatabáze je shodná s~datovými bloky obsaženými ve vstupním souboru VFK (pro každý datový blok je vytvořena jedna tabulka). 

Aplikace disponuje kromě zpracování souboru VFK dalšími funkcionalitami. Jednou z~hlavních je možnost kontroly struktury vstupního VFK souboru podle zvolené šablony nebo podle geodatabáze. Samozřejmostí je možnost uložení protokolu o~zpracování. Aplikace je distribuována zdarma.

\newpage
\subsection{ISKN View pro ArcGIS}
% TODO: https://www.arcdata.cz/produkty/ceska-specifika/iskn-pro-arcgis-for-desktop
ISKN View je sesterskou aplikací ke zvýše zmíněnému ISKN Studio (\ref{l_iskn_studio}). Software je používán jako doplněk (\textit{Add-In}) v~programu ArcGIS verze 10.1 a~vyšší. 

\begin{figure}[htb]
\centering
\includegraphics[scale=0.65]{images/ISKNView-aplikace.png}
\caption[Aplikace ISKN View]{Aplikace ISKN View (zdroj: \cite{iskn_studio})}
\end{figure}

Díky ISKN View je umožněno rychlé a~jednoduché vyhledávání v~datech ISKN, která byla pomocí aplikace ISKN Studio převedena do některé z~podporovaných geodatabází.  Aplikace je rovněž šířena bezúplatně. \cite{iskn_studio}

\newpage
\section{Import dat KN ve výměnném formátu}
% TODO: http://www.gisoft.cz/Moduly/ImportVFK
Jedná se o~modul vytvořený společností GISOFT, který slouží k~převodu a~načtení dat ve formátech VFK a~VKM do formátu DGN. Modul spolupracuje s~produkty společnosti Bentley Systems, především MicroStation. Umožňuje načtení dat jak ve starém, tak i~v~novém výměnném formátu (viz kapitola č. \ref{l_format_vfk}) KN. Je dostupný jako volitelný modul pro nadstavby \textbf{MGEO}\footnote{http://www.gisoft.cz/MGEO/MGEO} a~\textbf{SPIDER}\footnote{http://www.gisoft.cz/SPIDER/SPIDER}. Může být použit v~následujících případech:

\begin{description}
 \item[Samostatně:] Použití samostatně se hodí v~případě, kdy jsou načítána data výměnného formátu obsahující pouze katastrální mapu. V~tomto případě bude vstupní soubor převeden do podoby výkresu ve formátu DGN.

 \item[Spolu s~modulem Práce s~popisnými informacemi KN:] Tento mód je užiteč\-ný v~případě, kdy jsou v~souboru spolu s~katastrání mapou dostupné i~popisné informace (případně jsou uvedeny pouze popisné informace).	
\end{description}

\begin{figure}[htb]
\centering
\includegraphics[scale=1]{images/microstation_modul.png}
\caption[Import dat KN ve výměnném formátu -- ukázka použití]{Import dat KN ve výměnném formátu -- ukázka použití (zdroj: \cite{gisoft_modul})}
\end{figure}

Veškeré informace uvedené o~tomto modulu vycházejí z~oficiálního popisu modulu uvedeného na stránkách společnostu GISOFT. \cite{gisoft_modul}

\newpage
\section{Spirit VFK}
% TODO: http://www.georeal.cz/cz/spirit-desktop/spirit-vfk
Software Spirit VFK je vytvořený a~distribuovaný společností GEOREAL jako samostatně spustitelná desktopová aplikace. Slouží pro převod dat (VFK) katastru nemovitostí do jakékoli geodatabáze podporované společností ESRI. 

Do geodatabáze jsou postupně importovány tabulky, relace a~ostatní obsažené objekty ISKN. Takto vytvořená databáze může být použita v~souvisejících aplikačních nadstavbách \textbf{Spirit KN}\footnote{http://www.georeal.cz/cz/spirit-desktop/spirit-kn} a~\textbf{Spirit Portál - KN}\footnote{http://www.georeal.cz/cz/spirit-server/portal-kn}, případně může sloužit pro další analytické práce nad daty KN. Import dat do geodatabáze probíhá v~následujících krocích:

\begin{enumerate}
 \item příprava geodatabáze (tvorba tabulek, relací, \dots),
 \item import dat VFK,
 \item vektorizace parcel, budov a~ostatních mapových vrstev,
 \item optimalizace mapových vrstev, tvorba symbologie.
\end{enumerate}

Symbologie (ve formátech MXD a~LYR) vytvářená v~posledním kroku se používá pro zobrazování dat ve výše zmíněných aplikačních nadstavbách pro ArcMap. Aplikace Spirit VFK může být využívána pro pravidelnou aktualizaci datových skladů katastru nemovitostí.

\begin{figure}[htb]
\centering
\includegraphics[scale=0.9]{images/spirit_vfk.png}
\caption[Spirit VFK -- ukázka aplikace]{Spirit VFK -- ukázka aplikace (zdroj: \cite{spirit_vfk})}
\end{figure}

Pro aplikaci existuje i~její odlehčená verze Spirit VFK Light, díky které je možné importovat data VFK do osobní geodatabáze ArcGIS (MS Access) nebo databáze MS SQL Server. Používání obou aplikací nevyžaduje znalost struktury dat ISKN nebo výměnného formátu VFK.

Veškeré výše uvedené informace pocházejí z~oficiálních stránek produktu, viz \cite{spirit_vfk}.

\section{VFK2DWG}
\label{l_vfk2dwg}
% TODO: http://www.cadstudio.cz/vfk2dwg
Jedná se o~aplikaci od společnosti CAD Studio. Slouží jako nadstavba (utilita) pro produkty firmy Autodesk založených na AutoCadu (AutoCAD, AutoCAD Architecture, AutoCAD Map 3D, AutoCAD Civil 3D, \dots). Díky této aplikaci je možné do výše uvedených programů načíst data VF ISKN (\textit{.vfk}) a~dále s~nimi pracovat.

\begin{figure}[htb]
\centering
\includegraphics[scale=0.85]{images/vfk2dwg-aplikace.png}
\caption[Aplikace VFK2DWG]{Aplikace VFK2DWG (zdroj: \cite{cadstudio-vfk2dwg})}
\end{figure}

Aplikace převádí \textit{VFK} soubory na objekty (hranice parcel, parcelní čísla, vnitřní kresby, popisy, \dots), se kterými je AutoCAD schopen pracovat. Tyto objekty jsou pomocí hypertextových odkazů provázány se stránkami ČUZK (respektive s~aplikací \textbf{Nahlížení do KN}), kde o~nich mohou být zjištěny dodatečné informace.

% \begin{figure}[htb]
% \centering
% \includegraphics[width=\textwidth]{images/vfk2dwg-ukazka.png}
% \caption[Aplikace VFK2DWG -- ukázka načtených dat]{Aplikace VFK2DWG -- ukázka načtených dat (zdroj: \cite{cadstudio-vfk2dwg})}
% \end{figure}

Nejnovější verze aplikace pracuje i~s~daty formátu ve verzi 5.1 a~je podporována i~v~AutoCAD 2016. Bohužel se jedná o~komerční aplikaci a~proto nemohla být otestována. Veškeré informace byly převzaty z~oficiálních stránek společnosti CAD Studio, viz \cite{cadstudio-vfk2dwg}.


\section{VFK2DB}
% TODO: http://www.cadstudio.cz/vfk
VFK2DB je databázová varianta výše zmíněné aplikace (\ref{l_vfk2dwg}), která se chová jako samostatně spustitelný program nezávislý na konkrétním programu GIS či CAD. 

Aplikace importuje data z~formátu \textit{VFK} do relační databáze Oracle nebo MS SQL Server (v~budoucnu se počítá s~doplněním exportu do dalších typů databází, např. PostGIS, SQLite). Takto vytvořená databáze může být načtena některým z~GIS produktů založených na prostorových SQL databázích (AutoCAD Map 3D, ESRI, Bentley, Intergraph, GeoServer, MapServer, \dots). 

Opět se jedná o~komerční aplikaci společnosti CAD Studio, a~proto nemohla být otestována. Veškerý zde uvedený popis vychází z~oficiální dokumentace na stránkách společnosti, viz \cite{cadstudio-vfk2db}.

\section{Topol VFK Import}
% TODO: http://www.datasystem.cz/vfk-import-s-37-m-4.html
Jak už ze samotného názvu plyne, aplikace Topol VFK Import byla vyvinuta společností Data System s.r.o. ve spolupráci se společností Topol Software. Aplikace disponuje vlastním grafickým prostředím, ve kterém je možné VFK data exportovat do formátů DWG a~DXF, případně do vlastního formátu (OpenGIS MDB) společnosti Topol. 

Opět se jedná o~komerční aplikaci, a~proto nemohla být vykoušena. Veškeré informace pocházejí z~webových stránek výrobce, viz \cite{topol_vfk_import}.

\begin{figure}[htb]
\centering
\includegraphics[width=\textwidth]{images/topol-aplikace.png}
\caption[Topol VFK Import -- ukázka zpracovaných dat]{Topol VFK Import -- ukázka zpracovaných dat (zdroj: \cite{topol_vfk_import})}
\end{figure}

\newpage
\section{GDAL -- VFK Driver}
% TODO: http://freegis.fsv.cvut.cz/gwiki/VFK_/_GDAL
% TODO: http://www.gdal.org/drv_vfk.html
% TODO: http://gama.fsv.cvut.cz/~landa/publications/2010/gis-ostrava-2010/paper/landa-ogr-vfk.pdf
VFK Driver, díky kterému je možné data VFK číst, je součástí knihovny GDAL (viz \ref{l_gdal}) od verze 1.7. Vstupní VFK soubor je knihovnou GDAL rozeznán jako \texttt{OGR datasource}, každý datový blok je poté vnímán jako \texttt{OGR layer}. Od GDAL verze 1.10 je podpora VFK přidána pouze v~případě, kdy je knihovna kompilována s~podporou SQLite (\texttt{./configure --with-sqlite}).

Driver si interně data při prvním čtení ukládá do databáze SQLite ve stejném adresáři, jako je umístěn VFK soubor. Při opětovném čtení driver používá pro čtení již vytvořenou databázi. Tímto se opakované načítání dat několikanásobně urychlí, viz porovnání níže. Implicitní chování driveru může být ovlivněno zadáním proměnných prostředí.

\begin{multicols}{2}
\begin{lstlisting}
# prvni cteni
real	0m11.547s
user	0m11.016s
sys	0m0.232s
\end{lstlisting}
\columnbreak
\begin{lstlisting}
# opakovane cteni
real	0m0.317s
user	0m0.284s
sys	0m0.028s
\end{lstlisting}
\end{multicols}

Jednou z~nejdůležitějších je proměnná \texttt{OGR\_VFK\_DB\_NAME} sloužící pro definici jména SQLite databáze. Neméně důležitá proměnná
\texttt{OGR\_VFK\_DB\_OVERWRITE} říká, že při každém čtení souboru VFK se vytváří databáze SQLite znovu, čtení tedy probíhá pouze ze souboru. Níže je uvedena ukázka otevření souboru VFK. \cite{gdal_vfk}

\begin{lstlisting}[language=bash, showstringspaces=false, escapeinside={(*@}{@*)}]
$ ogrinfo Export_vse.vfk
Had to open data source read-only.
INFO: Open of `Export_vse.vfk'
      using driver `VFK' successful.
1: PAR (Polygon)
2: BUD (Polygon)
3: ZPOCHN (None)
(*@{\hspace{40pt}\raisebox{-1pt}[0pt][0pt]{$\vdots$}}@*)
74: BUDOBJ (None)
75: ADROBJ (None) 
\end{lstlisting}

\begin{figure}[htbp]
\centering
\includegraphics[width=\textwidth]{images/grass_ukazka.png}
\caption[Ukázka načtení VFK pomocí VFK Driveru GDAL v programu GRASS GIS]{Ukázka načtení VFK pomocí VFK Driveru GDAL v programu GRASS GIS (zdroj: vlastní)}
\end{figure}



\clearpage
\chapter{Použité technologie}

\section{QGIS}

QGIS je geografický informační systém, který je distribuován jako open-source\footnote{Open-source software je takový software, k~němuž zákazník dostane od jeho tvůrce zdrojový kód a~může jej dále upravovat. Jednotlivé definice termínu \uv{open source} se liší zvláště v~podmínkách pro další distribuci softwaru.\cite{abclinuxu_opensource}} pod licencí \textit{GNU General Public License}. Je oficiálním a~klíčovým produktem organizace OSGeo. Díky přenositelnosti zdrojového kódu je použitelný na širokém spektru platforem, ať už jsou to desktopové platformy Linux, MacOS, Windows, nebo mobilní platforma Android.

\begin{figure}[htb]
\centering
\includegraphics[scale=1]{images/qgis-logo.png}
\caption[QGIS -- logo]{QGIS -- logo (zdroj: \cite{qgis})}
\end{figure}

Program umožňuje prohlížení, tvorbu a~editaci velkého množství vektorových (Esri Shapefile, GeoJSON, GPX, \dots), ale i~rastrových (GeoTIFF, JPEG, \dots) nebo databázových formátů. Podporuje zpracování dat GPS a~tvorbu mapových výstupů. Mimo jiné umožňuje provádět prostorové analýzy, analýzy terénu nebo analýzy síťové, práci s~mapovou algebrou a~mnoho dalšího.

QGIS nedisponuje tak širokou paletou nástrojů, jako jeho open-source kolega GRASS GIS. Jeho funkcionalita ale může být rozšířena díky nepřebernému množství zásuvných modulů. Jedním z~nejdůležitějších modulů pro analýzu geografických dat je zásuvný modul GRASS GIS, který zpřístupňuje funkce stejnojmenného programu. QGIS poté může sloužit jako jeho nadstavba.
\cite{qgis}
\cite{qgis_wiki}


\section{GDAL/OGR}
\label{l_gdal}

GDAL je knihovna určená pro čtení a~zápis rastrových GIS formátů. Knihovna je vyvíjena pod hlavičkou Open Source Geospatial Foundation a~vydávána pod licencí \textit{X/MIT}. Knihovna používá jednoduchý abstraktní datový model pro všechny podporované datové formáty. Kromě toho nabízí také řadu užitečných nástrojů pro příkazovou řádku určených pro konverzi a~zpracování dat. \cite{gdal_wiki}

\begin{figure}[h]
\centering
\includegraphics[scale=1]{images/gdal-logo.png}
\caption[GDAL -- logo]{GDAL -- logo (zdroj: \cite{gdal})}
\end{figure}

GDAL byla původně vyvíjena Frankem Warmerdamem a~to do verze 1.3.2, posléze byla knihovna převedena na GDAL/OGR Project Management Committee, která je součástí Open Source Geospatial Foundation.\cite{gdal_wiki}

Knihovna OGR, která je od verze 2.0 součástí knihovny GDAL/OGR, slouží pro práci s~daty ve vektorovém formátu.\cite{gdal}

GDAL/OGR je považován za jeden z~hlavních open-source projektů. Knihovna je hojně využívána také v~komerční GIS sféře. Knihovna je otevřená a~poskytuje základní funkcionalitu potřebnou pro denní práci s~rozsáhlým množstvím GIS formátů.\cite{gdal_wiki}


\section{Python}

Jazyk Python je objektově orientovaný programovací jazyk, který efektivně používá víceúrovňové datové typy. Jedná se o~jazyk interpretovaný, čímž se jeví jako ideální nástroj pro psaní skriptů, ale i~rychlý vývoj aplikací. Je vyvíjen jako open-source software, díky čemuž se stává použitelným na velkém množství platforem (Linux, Windows, MacOS, \dots). Jazyk je rozšířitelný o~široké spektrum modulů, které umožňují řešit problematiku takřka z~jakékoli oblasti. V~současné době je Python vyvíjen ve dvou verzích, ve verzi 2.x a~v~novější verzi 3.x.
\cite{dive_into_python}
\cite{python_web}

\begin{figure}[htb]
\centering
\includegraphics[scale=1]{images/python-logo.png}
\caption[Python -- logo]{Python -- logo (zdroj: \cite{python_web})}
\end{figure}

\section{PyQt}

PyQt je modul, který zpřístupňuje knihovnu Qt pro programovací jazyk Python. Spolu s~PySide se jedná o~nejznámější a~nejpoužívanější modul pro Python postavený nad knihovnou Qt. Je vyvíjen britskou firmou Riverbank Computing ve dvou verzích. Ve verzi 4, podporující knihovnu Qt 4, a~ve verzi 5, která podporuje novější verzi Qt knihovny. Modul je dostupný na všech platformách, které podporují knihovnu Qt (Windows, MacOS/X a~Linux). PyQt je šířeno pod tzv. dvojí licencí, \textit{GNU GPL v3} a~\textit{Riverbank Commercial License}. Spolu s~těmito licencemi je dostupné i~pod komerční licencí.

\begin{figure}[htb]
\centering
\includegraphics[scale=1]{images/pyqt-logo.png}
\caption[PyQt -- logo]{PyQt -- logo (zdroj: \cite{pyqt_wiki})}
\end{figure}

Pro grafický návrh aplikace je vhodné použít nativní grafické uživatelské rozhraní Qt Designer. Výstupem z~tohoto programu je soubor obsahující vzhled aplikace ve formátu \textit{.xml}. PyQt je poté schopné tento formát převést do kódu jazyka Python. Pro komunikaci mezi objekty je využíváno signálů a~slotů, díky čemuž je vytvoření komponent velice snadné.

PyQt v~sobě kombinuje mocnost knihovny Qt s~jednoduchostí jazyka Python, což z~něj dělá výkonný nástroj pro vývoj grafických aplikací.
\cite{pyqt}
\cite{pyqt_wiki}


% ==================================================================================
\clearpage
\chapter{Informační systém katastru nemovitostí}
\label{l_iskn}

ISKN je integrovaný informační systém pro podporu výkonu státní správy katastru nemovitostí a~pro zajištění jeho uživatelských služeb. Obsahuje prostředky pro současné vedení souborů popisných informací (SPI) a~souborů geodetických informací (SGI). Dále jsou v~něm obsaženy prostředky pro podporu správních a~administrativních činností při vedení katastru nemovitostí a~pro správu dokumentačních fondů. \cite{iskn}

\begin{figure}[htb]
\centering
\includegraphics[scale=1]{images/cuzk-logo.png}
\caption[ČUZK -- logo]{ČUZK -- logo (zdroj: \cite{iskn})}
\end{figure}

\section{Historie a~vývoj}

Vývoj systému byl započat v~roce 1997 ve spolupráci se společností APP Czech s.r.o.\footnote{Dnes společnost funguje pod názvem NESS Czech s.r.o.}, která fungovala jako systémový integrátor a~dodavatel aplikačního programového vybavení. Dalšími společnostmi podílejícími se na vývoji ISKN byly Infinity, a.s., Compaq Computer s.r.o.\footnote{Dnes pod názvem HP.}, Oracle Czech, s.r.o., Bentley Systems, s.r.o., BEA Systems, s.r.o. \cite{iskn}

Systém byl nasazen do provozu v~září roku 2001, a~to na všech katastrálních pracovištích včetně centrály. Dolaďování a~převzetí závěrečných etap probíhalo v~roce 2002. V~témže roce byl dokončen audit systému. \cite{iskn}

Implementace ISKN plně nahradila dřívější způsob vedení katastru nemovitostí. ISKN integroval vedení a~správu katastru nemovitostí pod jediný informační systém společný pro všechna pracoviště katastrálních úřadů a~centrum. Toto vede k~tomu, že je možné zveřejňovat a~poskytovat aktuální data z~katastru nemovitostí prostřednictvím dálkového přístupu během několika málo minut, a~to z~celého území republiky. \cite{iskn}

Data jsou do systému ISKN ukládána pomocí Spatial Cartridge Option do databáze Oracle. Podpora vzdáleného přístupu k~datům pomocí sítě Internet je zajištěna pomocí BEA WebLogic. Systémový management využívá nástrojů CA Unicenter. \cite{iskn}

V~roce 2004 byla uzavřena nová smlouva se společností NESS Czech s.r.o. na rozvoj a~údržbu informačního systému v~letech 2004 -- 2006. V~tomto období byl zmodernizován především Dálkový přistup do katastru nemovitostí a~zavedena orientační mapa parcel. Důležitou inovací  bylo zavedení elektronické značky pro výpis z~katastru nemovitostí a~pro kopii katastrální mapy \footnote{Tento krok umožnil, aby tzv. \uv{ověřující} podle zákona č. 365/2000 Sb., o~informačních systémech veřejné správy, v~platném znění, mohli poskytovat ověřené výpisy z~katastru nemovitostí, převedené z~elektronické do listinné podoby. \cite{iskn}}. \cite{iskn}

Společnost NESS Czech s.r.o. poté v~dalších letech vyhrála několik veřejných zakázek týkajících se údržby a~rozvoje ISKN. Hlavním cílem bylo převedení decentralizovaného systému (107 lokálních databází replikovaných do centrální databáze) na centralizovaný systém, ve kterém byla data ISKN uložena pouze v~jedné databázi. Spolu s~touto úpravou byla změněna i~architektura z~původní client/server na třívrstvou architekturu. Architektura je postavena na platformě Oracle Forms/Reports 10g a~databázi Oracle 10g. Další změnou byl přechod na vyšší verzi Bentley nástroje pro správu prostorových dat. \cite{iskn}

ISKN byl nadále zlepšován. Za zmínku stojí především systém pro Dálkový přístup do katastru nemovitostí nebo zavedení možnosti získat informaci o~ukončení řízení pomocí SMS nebo e-mailové zprávy. \cite{iskn}


\section{Hlavní charakteristiky ISKN}

\subsection{Optimalizace uložení dat}

Díky zvolení jednotného datového modelu pro uložení popisných a~prostorových dat v~databázi Oracle spolu s~daty týkajících se správních řízení byla umožněna současná aktualizace popisných a~prostorových dat a~udržení jejich vzájemné konzistence. Pro optimalizaci byla také přijata koncepce samostatné evidence budov a~bezešvé digitální katastrální mapy. Od konce roku 2001 jsou uchovávány také veškerá historická data popisných a~prostorových dat, díky čemuž je možné sestavovat data do potřebných výstupů k~historickému datu od zavedení ISKN v~roce 2001. \cite{iskn}

\subsection{Optimalizace procesů při správě KN}

Do systému ISKN byla zavedena celá řada automatických kontrol pro proces zapsání změny do KN. Dále bylo umožněno převzetí aktuálních dat z~jiných registrů (např. registr obyvatel) a~ostatních informačních systémů. Postup provedení změny dat KN je následující: na základě návrhu je připraven budoucí stav, který je možné před jeho zplatněním zobrazit (SPI, SGI), případně v~něm provádět úpravy. Toto zajišťuje důkladnou kontrolu výsledného stavu katastru. Proces realizace změny je navíc zajištěn i~technicko-organizačními opatřeními (návrh změny a~kontrolu, včetně zplatnění provádí vždy jiná osoba dle přidělených uživatelských rolí). \cite{iskn}

Díky novým procesům ve zpracování dat/návrhů změn je možné částečné nabytí platnosti geometrického plánu s~automatizovanou změnou návrhu změny v~budoucím stavu. Nové procesy také umožňují aktualizaci dat katastru nemovitostí takovým způsobem, aniž by zamkly aktualizovaná data. Pouze se jimi řeší konflikty v~aktualizaci stejných dat.

Součástí ISKN je také jednotná centrální správa číselníků, která vnáší jednotnost do procesu zpracování změn na katastrálních úřadech. Tímto se rapidně zvyšuje konzistence a~kvalita datové základny. Některé z~centrálních číselníků nebo seznamů jsou přebírány z~externích datových zdrojů (např. číselníky územní identifikace, PSČ). \cite{iskn}

\subsection{Bezpečnost}

Vysoká bezpečnost ochrany dat je zajištěna kombinací hardwarových prostředků s~operačním systémem, databází a~vlastní aplikací ISKN. Nepřetržitý provoz je zajištěn pomocí technologie databázových a~aplikačních clusterů a~tím, že je celá infrastruktura zdvojena (primární a~záložní centrum). Do záložního centra jsou replikována veškerá data tak, aby byl v~případě náhlého výpadku primárního centra zajištěn nepřetržitý provoz ISKN. \cite{iskn}


\section{Poskytování dat}

Poskytování dat je umožněno na základě vyhlášky číslo 358/2013 Sb., o~poskytování údajů z~katastru nemovitostí. \cite{iskn}

\subsection{Poskytování dat dálkovým přístupem}

Na základě registrace je umožněno poskytování dat (zdarma, nebo za úplatu podle typu zákazníka) prostřednictvím sítě Internet. Výpisy z~KN a~snímky katastrální mapy mají povahu veřejných elektronických listin (jsou opatřeny elektronickou značkou) a~mohou být převedeny do podoby listinných veřejných listin. Tímto způsobem je v~současné době vyřizována více než třetina výstupů. \cite{iskn}

Více informací o~této metodě poskytování dat je spolu s~aplikací dostupných na stránkách ČUZK (\url{http://www.cuzk.cz/aplikace-dp/}).

\subsection{Poskytování dat ve výměnném formátu ISKN}

Data z~KN mohou být poskytována v~textovém souboru, který obsahuje záznamy v~pevně definované struktuře. Více informací o~tomto výměnném formátu je uvedeno v~kapitole č. \ref{l_format_vfk}.


% ==================================================================================
\clearpage
\chapter{Výměnný formát ISKN}
\label{l_format_vfk}

V~této kapitole je ve stručnosti popsána historie vývoje výměnného formátu ISKN spolu s~jeho popisem, ve kterém se věnuji především sekcím důležitým pro vývoj zásuvného modulu pro QGIS, který dokáže data v~tomto formátu zobrazit.

\section{Vývoj formátu}

Hlavním milníkem ve vývoji výměnného formátu bylo zavedení ISKN, viz. kapitola č. \ref{l_iskn}. Do této doby byly soubory SPI a~SGI ukládány odděleně, což se právě se zavedením ISKN změnilo. Tento krok vedl k~vytvoření nového výměnného formátu (NVF), který postupně nahrazoval starý výměnný formát (SVF). \cite{dp_landa}

\subsection{Výměnný formát KN před ISKN}

Tento formát je po zavedení nového formátu také nazýván \textit{starý výměnný formát}~--~\textbf{SVF}. Byl vytvořen roku 1996, kdy začala vznikat digitalizace SGI. Obsahuje dvě samostatné části:

\begin{enumerate}
 \item \textbf{SPI} -- Obsahuje informace o~vlastnících, parcelách a~nabývacích titulech. Byl distribuován ve dvou formátech:
 \begin{enumerate}
  \item soubory ve formátu \textit{.dbf}: Tento typ souboru byl dále dělen na další dvě části:
  
  \begin{enumerate}
   \item SPI bez jiných právních vztahů (bez JPV),
   \item SPI s~jinými právními vztahy (s~JPV).
  \end{enumerate}
  
  \item soubory ve formátu \textit{.txt}: SB, SC, SE
 \end{enumerate}
 
 Data byla poskytována v~následujících rozsazích: podle územní jednotky (katastrální území, obec, okres, ČR), dle výběru parcel, nebo na základě oprávněného subjektu (pouze ve formátu \textit{.txt}).
 
 Ve výše zmíněných formátech (\textit{.dbf, .txt}) jsou značné nesoulady. Ty jsou způsobeny hlavně neexistencí některých položek v~novém datovém modelu ISKN nebo jejich rozdílnou interpretací.

 \item \textbf{SGI} -- Jsou v~něm obsaženy informace o~poloze nemovitostí. 
 
 Data byla poskytována pro katastrální území, kde již byla provedena digitalizace.
\end{enumerate}

V~současné době je již oficiální podpora ukončena a~byl nahrazen právě novým výměnným formátem. \cite{svf_cuzk}

\subsection{Výměnný formát VF ISKN}

Formát je nazýván také jako \textit{nový výměnný formát} -- \textbf{NVF}. V~tomto formátu jsou obsaženy zároveň popisné i~grafické informace včetně dat o~řízení. Data jsou vytvářena ve dvou stavech:

\begin{itemize}
 \item \textbf{Stavová data} -- Data jsou vygenerována vzhledem ke konkrétnímu časovému okamžiku. Obsahují vždy kompletní data pro daný okamžik. Práce s~těmito daty je řešena v~původní verzi zásuvného modulu pro QGIS. 
 
 \item \textbf{Změnová data} -- Jsou v~nich obsaženy pouze změny za požadovaný časový úsek. Zpracováním a~zobrazením změnových dat v~programu QGIS se zabývá právě tato diplomová práce.
\end{itemize}

Data jsou poskytována v~následujících rozsazích:

\begin{itemize}
 \item územní jednotka (katastrální území, obec, okres, ČR),
 \item oprávněný subjekt,
 \item výběr parcel,
 \item výběr parcel polygonem v~mapě.
\end{itemize}

Do výměnného souboru je možné dle přání zákazníka vybrat libovolné kombinace datových skupin, viz tab. \ref{t_datove_skupiny}. \cite{dp_landa}

\begin{table}[htbp]
\centering
\caption[Datové skupiny VF ISKN]{Datové skupiny VF ISKN (zdroj: \cite{nvf_cuzk})}
\begin{tabular}{ll}
\toprule
\textbf{Název skupiny} & \textbf{Obsah} \\ 
\midrule
Nemovitosti & parcely a~budovy \\ 
Jednotky & bytové jednotky \\ 
Bonitní díly parcel & kódy BPEJ k~parcelám \\ 
Vlastnictví & \parbox{220pt}{listy vlastnictví, oprávněné subjekty a~vlastnické vztahy} \\ 
Jiné právní vztahy & ostatní právní vztahy kromě vlastnictví \\ 
Řízení & údaje o~řízení (vklad, záznam,…) a~listiny \\ 
Prvky katastrální mapy & katastrální mapy v~digitální podobě \\ 
BPEJ & hranice BPEJ včetně kódů \\ 
Geometrický plán & geometrické plány \\ 
Rezervovaná čísla & rezervovaná parcelní čísla a~čísla PBPP \\ 
Definiční body & definiční body parcel a~staveb \\ 
Adresní místa & adresní místa budov \\
\bottomrule
\end{tabular}
\label{t_datove_skupiny}
\end{table}

\newpage
\section{Struktura výměnného formátu ISKN}

Tato kapitola pojednává o~struktuře výměnného formátu ISKN. Nejsou zde popsány a~do detailu rozvedeny veškeré datové bloky formátu, ale pouze ty nejdůležitější prvky formátu vzhledem k~zásuvnému modulu pro QGIS. Podrobný popis formátu je dostupný v~oficiální dokumentaci (\cite{vfk_struktura}), ze které tato kapitola čerpá. Veškeré ukázky výměnného formátu jsou pořízeny z~testovacích dat dostupných na stránkách ČUZK (\cite{nvf_cuzk}).

Datový soubor \textit{.vfk} se skládá ze tří základních částí, které budou samostatně popsány na následujících řádcích této kapitoly:

\begin{itemize}
 \item hlavička \texttt{\&H},
 \item datové bloky \texttt{\&B},
 \item koncový znak \texttt{\&K}.
\end{itemize}

Datový soubor je vytvářen v~kódování češtiny dle ČSN ISO 8859-2 (ISO Latin2)\footnote{Ve výjimečných případech je možné použít kódování WIN1250. Toto kódování je použito i~v~souboru ve formátu XML verze 1.0.}. Desetinným oddělovačem je tečka (.). Datum a~čas je uveden ve tvaru ``03.06.1999 09:58:42''. Jednotlivé záznamy na řádcích jsou odděleny pomocí středníku (;). Každá datová věta je ukončena pomocí souslednosti znaků \texttt{<CR><LF>}. Znak \uv{\texttt{\currency}} znamená, že následující řádek souboru výměnného formátu je pokračováním předchozího řádku a~tvoří jedinou datovou větu, která v~textové položce obsahuje formátovací znaky \texttt{<CR><LF>}. \cite{vfk_struktura}

\subsection{Hlavička \texttt{\&H}}

Každý řádek hlavičky začíná sousledností znaků \texttt{\&H}, po které následuje označení položky, např. \texttt{VERZE}. Jednotlivé údaje jsou odděleny pomocí středníku. Hlavička obsahuje několik povinných řádků, jejichž seznam je uveden v~tabulce \ref{t_hlavicka}.

\begin{table}[htbp]
\centering
\caption[Seznam položek hlavičky]{Seznam položek hlavičky (zdroj: \cite{vfk_struktura})}
\begin{tabular}{ll}
\toprule
\textbf{Položka} & \textbf{Popis} \\ 
\midrule
VERZE & označení verze VF \\ 
VYTVORENO & datum a~čas vytvoření souboru \\ 
PUVOD & původ dat \\ 
CODEPAGE & označení kódové stránky \\ 
SKUPINA & seznam skupin datových bloků souboru \\ 
JMENO & jméno osoby, která soubor vytvořila \\ 
PLATNOST & časová podmínka použitá pro vytvoření souboru \\ 
ZMENY & stavová, nebo změnová data \\ 
POLYG & omezující podmínka -- polygon \\
KATUZE & omezující podmínka -- katastrální území \\
OPSUB & omezující podmínka -- oprávněné subjekty \\
PAR & omezující podmínka -- parcely \\
\bottomrule
\end{tabular}
\label{t_hlavicka}
\end{table}

Příklad prvních řádků hlavičky je uveden v~tabulce \ref{t_hlavicka_priklad}. Tabulka byla vytvořena na základě testovacích dat a~obsahuje i~některé nepovinné položky, např. \texttt{\&HINFO}.

\begin{table}[htbp]
\centering
\caption[Ukázka hlavičky]{Ukázka hlavičky (zdroj: \cite{vfk_struktura})}
\begin{tabular}{ll}
\toprule
\textbf{Položka} & \textbf{Atributy} \\ 
\midrule
\&HVERZE & "5.0" \\ 
\&HINFO & "TESTOVACÍ" \\ 
\&HVYTVORENO & "23.11.2013 12:58:06" \\ 
\&HPUVOD & "ISKN" \\ 
\&HCODEPAGE & "WE8ISO8859P2" \\ 
\&HSKUPINA & "NEMO";"JEDN";"BDPA";"VLST";"JPVZ" \\ 
\&HJMENO & "Kokeš Petr Ing." \\ 
\&HPLATNOST & "23.11.2013 12:51:00";"23.11.2013 12:51:00" \\ 
\&HZMENY & 0 \\ 
\&HNAVRHY & 0 \\ 
\&HPOLYG & 0 \\ 
\bottomrule
\end{tabular}
\label{t_hlavicka_priklad}
\end{table}

\begin{description}
 \item[VERZE:] Pouze jeden řádek označující verzi souboru VFK.
 \item[VYTVOŘENO:] Datum a~čas, kdy byl datový soubor vygenerován.
 \item[PŮVOD:] Specifikuje původ dat. Standardně je zde uvedeno \uv{ISKN}.
 \item[CODEPAGE:] Označení kódové stránky. Hodnota \uv{WE8ISO8859P2} značí kódování češtiny dle ČSN ISO 8859-2. Hodnota \uv{"EE8MSWIN1250} slouží pro označení kódování češtiny dle MS WIN1250.
 \item[SKUPINA:] Uvádí se zde seznam datových bloků souboru. Např. \texttt{\&HSKUPINA; ”Zkratka skupiny“;[“Zkratka skupiny” \dots]}.
 \item[JMÉNO:] Jméno osoby, která soubor vytvořila. Např. \texttt{\&HJMENO;"Jméno Příjmení"}.
 \item[PLATNOST:] Časová podmínka použitá pro vytvoření souboru. Zde jsou možné dvě varianty:
 
 \begin{itemize}
  \item Data jsou platná v~daném čase. \texttt{\&HPLATNOST;"03.12.2013 09:56:42"; "03.12.2013 09:56:42"},
  \item data jsou platná v~daném období. \texttt{\&HPLATNOST;"03.12.2012 09:56:42"; "03.12.2013 09:56:42"}.
 \end{itemize}
 
 S~tímto souvisí položka \texttt{\&HZMENY}, která nabývá hodnot 0/1 a~označuje, zda se jedná o~data stavová, nebo změnová. Položka \texttt{\&HNAVRHY} nabývá také hodnot 0/1 a~značí, zda jsou v~souboru obsaženy potvrzené geometrické plány, či nikoliv.
 
 \item[KATUZE:] Obsahuje jeden řádek, který popisuje hlavičku omezující podmínky katastrálních území. Další řádky začínající \texttt{\&D} tvoří omezující podmínku. Počet datových řádků udává počet katastrálních území, která omezující podmínku tvoří. Pokud v~omezující podmínce není žádné katastrální území, bude uvedena pouze hlavička. Pro ujasnění je zde uveden příklad z~testovacích dat.
 
 \begin{lstlisting}
&HKATUZE;KOD N6;OBCE_KOD N6;NAZEV T48;PLATNOST_OD D;PLATNOST_DO
&DKATUZE;693936;550426;"Jama";"19.06.1991 00:00:00";""
 \end{lstlisting}

 \item[OPSUB:] První řádek popisuje hlavičku omezující podmínky oprávněných subjektů. Další řádky s~daty poté omezující podmínku tvoří, obdobně jako je uvedeno u~omezující podmínky pro katastrální území. Počet datových řádků je shodný s~počtem oprávněných subjektů v~omezující podmínce.
 
 \item[PAR:] První řádek popisuje hlavičku omezující podmínky parcel. Další řádky tvoří omezující podmínku. Počet datových řádků je shodný s~počtem parcel uvedených v~omezující podmínce.
 
 \item[POLYG:] Tento údaj může nabývat hodnot 0/1. Pokud je uvedena hodnota 1, tak je obsah souboru odvozen z~polygonu. V~takovém to případě musí být polygon na dalších řádcích definován svými vrcholy. Takto zadaný polygon může mít nejvýše 101 vrcholů. Příklad zadání omezujícího polygonu:
 
 \begin{lstlisting}
&HPOLYGDATA;675124.12;1024587.24
&HPOLYGDATA;675224.12;1024687.24
&HPOLYGDATA;675184.12;1024537.24
 \end{lstlisting} 
\end{description}

\subsection{Datové bloky}

Datové bloky obsahují řádky dvojího typu:

\begin{itemize}
 \item uvozující řádek bloku \texttt{\&B} -- obsahuje seznam atributů s~jejich datovými typy, viz tab. \ref{t_datove_typy},
 \item datové řádky \texttt{\&D} -- v~řádku jsou uvedeny vlastní data.
\end{itemize}

\begin{table}[htbp]
\centering
\caption[Datové typy ISKN]{Datové typy ISKN (zdroj: \cite{dp_landa})}
\begin{tabular}{lll}
\toprule
\textbf{Kód} & \textbf{Datový typ} & \textbf{Číslo za kódem} \\ 
\midrule
N & číselný & maximální délka položky \\ 
T & textový & maximální délka textu \\ 
D & datumový & ve tvaru DD.MM.YYYY HH:MI:SS \\ 
\bottomrule
\end{tabular}
\label{t_datove_typy}
\end{table}

Níže je uveden příklad datového bloku pro blok \uv{PARCELA}. Ukázka je pořízena z~testovacích dat.

\begin{lstlisting}
&BPAR;ID N30;STAV_DAT N2;DATUM_VZNIKU D;DATUM_ZANIKU D;PRIZNAK_KONTEXTU N1;RIZENI_ID_VZNIKU N30;RIZENI_ID_ZANIKU N30;PKN_ID N30;PAR_TYPE T10;KATUZE_KOD N6;KATUZE_KOD_PUV N6;DRUH_CISLOVANI_PAR N1;KMENOVE_CISLO_PAR N5;ZDPAZE_KOD N1;PODDELENI_CISLA_PAR N3;DIL_PARCELY N1;MAPLIS_KOD N30;ZPURVY_KOD N1;DRUPOZ_KOD N2;ZPVYPA_KOD N4;TYP_PARCELYN1;VYMERA_PARCELY N9;CENA_NEMOVITOSTI N14.2;DEFINICNI_BOD_PAR T100;TEL_ID N30;PAR_ID N30;BUD_ID N30;IDENT_BUD T1;SOUCASTI T1;PS_ID N30;IDENT_PS T1

&DPAR;3067989306;0;"26.06.2003 07:43:05";"";3;3003873306;;;"PKN"; 693936;;1;37;;1;;6780;2;13;;;332;;"";674674306;;323700306;"a";"n";;"n"
\end{lstlisting}

\subsection*{Seznam skupin datových bloků ISKN}

V~této sekci je uveden popis jednotlivých skupin datových bloků. Jsou zde uvedeny pouze nejpodstatnější informace, podrobný popis lze dohledat v~oficiální dokumentaci formátu VFK (\cite{vfk_struktura}).

\begin{description}
 \item[NEMOVITOSTI:] Jedná se o~největší skupinu datových bloků. Celkem jich může obsahovat až 21. V~této skupině se nachází dva nejdůležitější bloky z~pohledu zásuvného modulu pro QGIS, a~to bloky PAR a~BUD. Právě tyto dva bloky jsou pomocí zásuvného modulu vizualizovány. Dále je zde obsažen například číselník způsobů využití pozemku nebo způsob využití budov. Seznam všech bloků v~této skupině je uveden v~tabulce \ref{t_skupina_nemovitosti}. Znak (*) uvedený v~tabulce znamená, že daný blok nepodléhá historizaci.
 
\begin{table}[htbp]
\centering
\caption[Seznam datových bloků ve skupině \uv{NEMOVITOSTI}]{Seznam datových bloků ve skupině \uv{NEMOVITOSTI} (zdroj: \cite{vfk_struktura})}
\begin{tabular}{ll}
\toprule
\textbf{Kód} & \textbf{Popis} \\ 
\midrule
PAR & Parcely \\ 
BUD & Budovy \\ 
CABU & Části budov \\ 
ZPOCHN* & Číselník způsobů ochrany nemovitosti \\
DRUPOZ* & Číselník druhů pozemku \\
ZPVYPO* & Číselník způsobů využití pozemku \\
ZDPAZE* & Číselník zdrojů parcel ZE \\
ZPURVY* & Číselník způsobů určení výměry \\
TYPBUD* & Číselník typů budov \\
MAPLIS* & Číselník mapových listů \\
KATUZE* & Číselník katastrálních území \\
OBCE* & Číselník obcí -- vázaně \\
CASOBC* & Číselník částí obce -- vázaně \\
OKRESY* & Číselník okresů -- vázaně \\
KRAJE* & Číselník krajů -- vázaně \\
NKRAJE* & Číselník nových krajů -- vázaně \\
RZO & Přiřazení způsobu ochrany k~nemovitostem \\
ZPVYBU* & Způsob využití budov \\
PS & Práva stavby \\
RU & Přiřazení účelu práva stavby \\
UCEL & Číselník účelů práva stavby \\
\bottomrule
\end{tabular}
\label{t_skupina_nemovitosti}
\end{table}
 
 \item[JEDNOTKY:] V~této skupině jsou uvedeny bytové či nebytové prostory, které byly označeny příslušnou listinou jako jednotka. Pro každou jednotku je uveden její popis (jednoznačně ji identifikuje v~rámci budovy), typ a~způsob využití. Ke každé jednotce je dále uveden spoluvlastnický podíl ($\frac{\text{velikost podlahové plochy}}{\text{celková plocha všech jednotek v~domě}}$). \cite{dp_landa}
 
 
\begin{table}[htbp]
\centering
\caption[Seznam datových bloků ve skupině \uv{JEDNOTKY}]{Seznam datových bloků ve skupině \uv{JEDNOTKY} (zdroj: \cite{vfk_struktura})}
\begin{tabular}{ll}
\toprule
\textbf{Kód} & \textbf{Popis} \\ 
\midrule
JED & Jednotky \\ 
TYPJED* & Číselník typů jednotek \\ 
ZPVYJE* & Způsob využití jednotek \\ 
\bottomrule
\end{tabular}
\label{t_skupina_jednotky}
\end{table}

 \newpage
 \item[BONITNÍ DÍLY PARCEL:] Jsou zde uvedeny informace o~bonitních dílech parcely. Ve skupině se nachází pouze jeden datový blok (BDP), ve kterém je popsán vztah mezi BPEJ\footnote{\textbf{Bonitovaná půdně-ekologická jednotka}: Základní určovací a~oceňovací jednotka produkční schopnosti zemědělské půdy. Je vyjádřená číselným kódem -- číslice kódu vyjadřují půdně-klimatické vlastnosti půdy. Jednotky tvoří ohraničený územní celek, který má specifické ekologické vlastnosti a~bioenergetický potenciál. \cite{vugtk}} a~parcelou. \cite{dp_landa}
 
 \item[VLASTNICTVÍ:] Tato skupina bloků obsahuje informace o~vlastnictví. Jako vlastník zde může být uvedena fyzická osoba, právnická osoba nebo jiný oprávněný uživatel (manželé v~bezpodílovém spoluvlastnictví). Ve skupině se může nacházet několik datových bloků, viz tab. \ref{t_skupina_vlastnictvi}. \cite{dp_landa}
 
\begin{table}[htbp]
\centering
\caption[Seznam datových bloků ve skupině \uv{VLASTNICTVÍ}]{Seznam datových bloků ve skupině \uv{VLASTNICTVÍ} (zdroj: \cite{vfk_struktura})}
\begin{tabular}{ll}
\toprule
\textbf{Kód} & \textbf{Popis} \\ 
\midrule
OPSUB & Oprávněné subjekty \\ 
VLA & Vlastnictví \\ 
CHAROS* & Číselník charakteristik oprávněných subjektů \\ 
TEL & Katastrální tělesa \\
\bottomrule
\end{tabular}
\label{t_skupina_vlastnictvi}
\end{table}

 \item[JINÉ PRÁVNÍ VZTAHY:] Obsahuje informace o~jiných než vlastnických vztazích jednoho oprávněného subjektu (nemovitosti) ke konkrétnímu předmětu (nemovitosti, vlastnictví, dalšímu jinému právnímu vztahu). Seznam a~popis datových bloků je uveden v~tabulce č. \ref{t_skupina_jpv}. \cite{dp_landa}
 
\begin{table}[htbp]
\centering
\caption[Seznam datových bloků ve skupině \uv{JINÉ PRÁVNÍ VZTAHY}]{Seznam datových bloků ve skupině \uv{JINÉ PRÁVNÍ VZTAHY} (zdroj: \cite{vfk_struktura})}
\begin{tabular}{ll}
\toprule
\textbf{Kód} & \textbf{Popis} \\ 
\midrule
JPV & Jiné právní vztahy \\ 
TYPRAV* & Číselník typů právních vztahů \\ 
RJPV & Vazba JPV k~jinému věcnému právu \\ 
\bottomrule
\end{tabular}
\label{t_skupina_jpv}
\end{table}

 \item[ŘÍZENÍ:] Tato skupina je druhou nejobsáhlejší skupinou ve výměnném formátu. Může být obsažena ve změnovém exportu z~ISKN -- v~tomto případě obsahuje pouze záznamy, které byly v~daném časovém intervalu změněny. Obsahuje několik datových bloků, viz tab. \ref{t_skupina_rizeni}. \cite{dp_landa}
 
\begin{table}[htbp]
\centering
\caption[Seznam datových bloků ve skupině \uv{ŘÍZENÍ}]{Seznam datových bloků ve skupině \uv{ŘÍZENÍ} (zdroj: \cite{vfk_struktura})}
\begin{tabular}{ll}
\toprule
\textbf{Kód} & \textbf{Popis} \\ 
\midrule
RIZENI* & Řízení (vklad, záznam) \\ 
RIZKU* & Vazba Řízení -- Katastrální území \\ 
OBJRIZ* & Objekty řízení (parcely, budovy, \dots) \\ 
PRERIZ* & Předměty řízení \\ 
UCAST* & Účastníci řízení \\ 
ADRUC* & Adresy účastníků řízení \\ 
LISTIN* & Listiny \\ 
DUL* & Další údaje listin \\ 
LDU* & Vazba Listiny -- Další údaje listin \\ 
TYPLIS* & Číselník typů listin \\ 
TYPPRE* & Číselník typů předmětu řízení \\ 
TYPRIZ* & Typy řízení \\ 
TYPUCA* & Typy účastníků řízení \\ 
UCTYP* & Vazba Účastnící -- Typy účastníků řízení \\ 
RL & Přiřazení listin k~nemovitostem, vlastnictví a~jiným právním vztahům \\ 
OBESMF* & Obeslání účastníků řízení \\ 
\bottomrule
\end{tabular}
\label{t_skupina_rizeni}
\end{table}
 
 \item[PRVKY KATASTRÁLNÍ MAPY:] Tato skupina je jednou z~nejdůležitějších pro zásuvný modul, respektive pro driver VFK v~knihovně GDAL. Jsou v~ní totiž obsaženy jak popisné, tak hlavně polohopisné informace o~prvcích polohopisu. Z~nich (tedy hlavně z~prvních dvou datových bloků \texttt{SOBR} a~\texttt{SBP}) je tvořena geometrie vektorové mapy. Spolu s~nimi skupina obsahuje další neméně důležité datové bloky, viz tab. \ref{t_skupina_prvkyKM}. \cite{dp_landa}

\begin{table}[htbp]
\centering
\caption[Seznam datových bloků ve skupině \uv{PRVKY KATASTRÁLNÍ MAPY}]{Seznam datových bloků ve skupině \uv{PRVKY KATASTRÁLNÍ MAPY} (zdroj: \cite{vfk_struktura})}
\begin{tabular}{ll}
\toprule
\textbf{Kód} & \textbf{Popis} \\ 
\midrule
SOBR* & Souřadnice obrazu bodů polohopisu v~mapě \\ 
SBP & Spojení bodů polohopisu -- definuje polohopisné liniové prvky \\ 
SBM & Spojení bodů mapy -- definuje nepolohopisné liniové prvky \\ 
KODCHB* & Číselník kódů charakteristiky kvality bodu \\ 
TYPSOS* & Číselník typů souřadnicových systémů \\ 
HP & Hranice parcel \\ 
OP & Obrazy parcel (parcelní číslo, značka druhu pozemku, \dots) \\ 
OB & Obrazy budov (obvod budovy, značka druhu budovy) \\ 
DPM & Další prvky mapy \\ 
OBBP & Obrazy bodů BP \\ 
TYPPPD* & Číselník typů prvků prostorových dat \\ 
ZVB & Zobrazení věcných břemen \\ 
POM & Prvky orientační mapy \\ 
SPOM & Spojení prvků orientační mapy -- definuje liniové prvky \\ 
SPOL & Souřadnice polohy bodů polohopisu (měřené) \\  
\bottomrule
\end{tabular}
\label{t_skupina_prvkyKM}
\end{table}
 
 \newpage
 \item[BPEJ:] Skupina \texttt{BPEJ} obsahuje dva datové bloky, viz tab. \ref{t_skupina_bpej}. Jsou v~ní obsaženy informace o~bonitovaných půdně ekologických jednotkách. BPEJ je základní mapovací a~oceňovací jednotka zemědělských půd, která vyjadřuje rozdílné produkční a~ekonomické efekty zemědělského území. Hranice BPEJ nejsou součástí katastrální mapy. Pouze tvoří rozhraní mezi dvěma jednotkami. \cite{dp_landa}

\begin{table}[htbp]
\centering
\caption[Seznam datových bloků ve skupině \uv{BPEJ}]{Seznam datových bloků ve skupině \uv{BPEJ} (zdroj: \cite{vfk_struktura})}
\begin{tabular}{ll}
\toprule
\textbf{Kód} & \textbf{Popis} \\ 
\midrule
HBPEJ & Hranice BPEJ \\ 
OBPEJ & Označení BPEJ \\ 
\bottomrule
\end{tabular}
\label{t_skupina_bpej}
\end{table}
 
 \item[GEOMETRICKÝ PLÁN:] Tato skupina obsahuje sadu bloků popisujících geometrický plán a~hlavičku dalších změn v~KM, které nejsou prováděny geometrickým plánem. Obsahuje několik datových bloků, jejichž seznam je uveden v~tabulce č. \ref{t_skupina_gp}. Ve skupině je obsažena tabulka pro uchování záznamů podrobného měření změn jak v~terénu, tak i~změn, které s~měřením v~terénu nesouvisí (slučování parcel, demolice budov, \dots). \cite{dp_landa} \cite{vfk_struktura}

\begin{table}[htbp]
\centering
\caption[Seznam datových bloků ve skupině \uv{GEOMETRICKÝ PLÁN}]{Seznam datových bloků ve skupině \uv{GEOMETRICKÝ PLÁN} (zdroj: \cite{vfk_struktura})}
\begin{tabular}{ll}
\toprule
\textbf{Kód} & \textbf{Popis} \\ 
\midrule
NZ & Hlavičky geometrických plánů a~ostatních změn KM \\ 
ZPMZ & Hlavičky ZPMZ \\ 
NZZP & Vazební tabulka návrhy změn KM -- ZPMZ \\ 
PARG & Parcely GP \\ 
BUDG & Budovy GP \\ 
BDPG & Bonitní díly parcel GP \\ 
HPG & Hranice parcel GP \\ 
OPG & Obrazy parcel GP \\ 
OBG & Obrazy budov GP \\ 
ZVBG & Zobrazení věcných břemen GP \\ 
DPMG & Další prvky mapy GP \\ 
SBPG & Spojení bodu polohopisu GP \\ 
OBPEJG & Označení BPEJ GP \\ 
SBMG & Spojení bodů mapy GP \\ 
HBPEJG & Hranice BPEJ GP \\ 
OBDEBOG & Obrazy definičních bodů parcel a~budov GP \\ 
\bottomrule
\end{tabular}
\label{t_skupina_gp}
\end{table}

 \newpage
 \item[REZERVOVANÁ ČÍSLA:] Rezervovanými čísly se v~této skupině myslí parcelní čísla, která byla rezervována pro účely vyhotovení geometrického plánu. Před potvrzením geometrického plánu probíhá kontrola, jestli byla použita přidělená rezervovaná parcelní čísla. Při zápisu nové parcely do KN se její číslo z~tabulky \texttt{RECI} maže. Úplné parcelní číslo musí být jedinečné v~rámci tabulek \texttt{PAR} a~\texttt{RECI}. 
 
 Datový blok \texttt{DOCI} obsahuje všechna parcelní čísla, která kdy byla použita za dobu elektronického vedení katastru nemovitostí v~informačním systému ISKN (od r. 2001). Seznam všech datových bloků je uveden v~tabulce č. \ref{t_skupina_rezCisla}. \cite{dp_landa} \cite{vfk_struktura}

\begin{table}[htbp]
\centering
\caption[Seznam datových bloků ve skupině \uv{REZERVOVANÁ ČÍSLA}]{Seznam datových bloků ve skupině \uv{REZERVOVANÁ ČÍSLA} (zdroj: \cite{vfk_struktura})}
\begin{tabular}{ll}
\toprule
\textbf{Kód} & \textbf{Popis} \\ 
\midrule
RECI & Rezervovaná parcelní čísla \\
DOCI & Dotčená parcelní čísla \\
DOHICI & Dotčená historická parcelní čísla \\
REZBP & Rezervovaná čísla bodu PBPP \\
\bottomrule
\end{tabular}
\label{t_skupina_rezCisla}
\end{table}

 \item[DEFINIČNÍ BODY:] Skupina obsahuje pouze jeden datový blok \texttt{OBDEBO}. V~tomto bloku jsou obsaženy obrazy definičních bodů parcel, budov a~částí budov (pokud jsou v~ISKN naplněny). Jsou zde uvedeny údaje o~souřadnicích a~odkazech (ID) na objekty v~KN. \cite{vfk_struktura}
 
 \item[ADRESNÍ MÍSTA:] Datový blok \texttt{BUDOBJ} zajišťuje vazbu mezi budovami a~adresami pomocí ID budovy a~kódu objektu. Tento blok nepracuje s~historií -- obsahuje vždy aktuální data bez ohledu na datum, ke kterému je export NVF vytvořen.
 
 Ve druhém bloku (\texttt{ADROBJ}) jsou uvedeny odkazy na adresy budov, které jsou obsaženy v~bloku nemovitostí. Blok opět nepracuje s~historií. \cite{vfk_struktura}

\begin{table}[htbp]
\centering
\caption[Seznam datových bloků ve skupině \uv{ADRESNÍ MÍSTA}]{Seznam datových bloků ve skupině \uv{ADRESNÍ MÍSTA} (zdroj: \cite{vfk_struktura})}
\begin{tabular}{ll}
\toprule
\textbf{Kód} & \textbf{Popis} \\ 
\midrule
BUDOBJ & Odkazy objektů na adresy \\
ADROBJ & Adresy \\
\bottomrule
\end{tabular}
\label{t_skupina_adrMista}
\end{table}
 
\end{description}


\subsection{Koncový znak \texttt{\&K}}

Specifickou částí výměnného formátu je takzvaný \uv{koncový znak} \texttt{\&K}. Načtení tohoto znaku signalizuje konec souboru výměnného formátu. Pro driver VFK v~knihovně GDAL znak znamená konec načítání. 


\section{Změnové věty v~NVF}

Změnový export nelze provést nad všemi skupinami datových bloků. Proveden může být pouze nad následujícími:

\begin{itemize}
\itemsep 1pt	
\parskip 1pt
 \item nemovitosti,
 \item jednotky,
 \item bonitní díly parcel,
 \item vlastnictví,
 \item JPV,
 \item řízení,
 \item prvky katastrální mapy,
 \item BPEJ.
\end{itemize}

Objekty, které byly vybrány v~parametrech při spuštění exportu, jsou součástí datového souboru i~v~případě, kdy na nich nebyla provedena žádná změna (tzn. jsou platné). Pro ostatní objekty ve vybraných datových skupinách se exportují pouze změnové věty. \cite{vfk_struktura}

\subsection{Obsah změnového exportu -- Typy tabulek}

Z~pohledu exportu změnových vět rozlišujeme několik skupin tabulek, které jsou rozděleny podle toho, zda podléhají principu historizace, či nikoliv.

\subsubsection{Tabulky předmětu KN podléhající principu historizace}

Uchovává se u~nich jak minulost, tak současnost. Tento typ obsahuje tabulky, ve kterých jsou uloženy informace o~parcelách, budovách, jednotkách,  OS, JPV, přiřazených listinách a~katastrálních tělesech. Aktuálnost dat je vyjádřena pomocí následujících atributů: datum vzniku, datum zániku, stav dat a~kontext změn. Může nastat několik kombinací atributů, jejich seznam a~popis je uveden v~tabulce č. \ref{t_komb_atributu}. \cite{vfk_struktura}

\begin{table}[htbp]
\centering
\caption[Kombinace atributů vyjadřujících aktuálnost dat]{Kombinace atributů vyjadřujících aktuálnost dat (zdroj: \cite{vfk_struktura})}
\label{t_komb_atributu}
\begin{tabular}{lccl}
\toprule
\textbf{Operace} & \textbf{\parbox{25pt}{Stav dat}} & \textbf{\parbox{50pt}{Kontext změn}} & \textbf{Událost} \\ \midrule
\multirow{3}{*}{UPDATE} & -1 & 1 & \parbox{245pt}{Objekt byl v~exportovaném období změněn, původní verze objektu zanikla, nová verze vznikla. Nová verze není vzhledem k~sys. datu aktuální, (záznam je v~minulosti).} \vspace{6pt} \\ 
 & -1 & 3 & \parbox{245pt}{Objekt v~exportovaném období vznikl a~později byl změněn -- verze není vzhledem k~sys. datu aktuální, (záznam je v~minulosti).} \vspace{6pt} \\ 
 & 0 & 3 & \parbox{245pt}{Objekt byl v~exportovaném období změněn, vznikla nová verze, která vzhledem k~sys. datu je aktuální.} \vspace{6pt} \\ 
DELETE & 3 & 1 & \parbox{245pt}{Objekt byl v~exportovaném období zrušen.} \vspace{6pt} \\ 
INSERT & 0 & 3 & \parbox{245pt}{Objekt vznikl v~exportovaném období.} \vspace{6pt} \\ 
LOCK & 0 & 2 & \parbox{245pt}{Objekty zadané ve vstupních parametrech.} \\ \bottomrule
\end{tabular}
\end{table}

Jestliže je provedena operace \texttt{UPDATE}, tak je možné mít v~exportovaném datovém souboru několik vět se stejným ID. Počet těchto vět je ovlivněn především počtem změn na daném objektu v~exportovaném období, ale i~vzájemným vztahem datových položek \textit{platnost od} a~\textit{platnost do} u~exportovaného objektu vzhledem k~danému období. 

U~operací \texttt{INSERT} a~\texttt{DELETE} je v~datovém souboru možný pouze jeden záznam s~jedním ID. \cite{vfk_struktura}

Příklad obsahu změnového exportu:

\begin{lstlisting}
ID N30;STAV_DAT N2;DATUM_VZNIKU D;DATUM_ZANIKU D;PRIZNAK_KONTEXTU N1;RIZENI_ID_VZNIKU N30;RIZENI_ID_ZANIKU N30
493589708;-1;"11.12.1998";"13.09.2002";1;908105708;919198708
493589708;-1;"13.09.2002";"14.11.2002";3;919198708;920435708
493589708;-1;"14.11.2002";"15.11.2002";3;920435708;920595708
493589708;-1;"15.11.2002";"";3;920595708;922200708
\end{lstlisting}


\subsubsection{Tabulky nepodléhající principu historizace (skupina RIZENI)}

Jsou exportovány daná řízení, která byla v~zadaném intervalu (Platnost od -- Platnost do) zplatněna, nebo uzavřena. Na tato řízení navazuje vazba na k.ú., objekty řízení, předměty říz., účast. říz., adresy, listiny, vazba na další údaje listin, typy účastníků řízení a~denormalizovaná data o~obeslání účastníků. \cite{vfk_struktura}

\subsubsection{Tabulky nepodléhají principu historizace (skupiny GMPL a~REZE)}

U~těchto tabulek se udržuje aktuální stav. Nemohou být proto obsahem změnových vět. Jsou zde udržovány informace o~geometrických plánech (skupina GMPL) a~rezervovaných číslech (skupina REZE). \cite{vfk_struktura}

\subsubsection{Export číselníků}

V~číselnících jsou exportována jen ta data, u~kterých byla platnost započata nebo ukončena v~zadaném časovém intervalu (Platnost od -- Platnost do). \cite{vfk_struktura}


% ==================================================================================
\clearpage
\chapter{VFK Plugin pro QGIS}
Prvotní verze zásuvného modulu QGIS \textbf{VFK Plugin} pro práci s daty českého katastru nemovitostí byla vyvinuta studenty oboru Geoinformatika na FSv ČVUT v Praze Annou Kratochvílovou a Václavem Petrášem. 

Zásuvný modul byl napsát v jazyce C++, s použitím knihovny Qt. Pracuje s daty v novém výměnném formátu katastru (VFK), viz kapitola č. \ref{l_format_vfk}. 

Pro přístup k datům plugin využívá knihovny GDAL, respektive ovladače \textbf{VFK Driver}. Načtená data driver ukládá do databáze SQLite, jejíž struktura je shodná se strukturou jednotlivých bloků v souboru VFK. Při opakovaném čtení dat ze stejného souboru je využívána již vytvořená SQLite databáze, oproti původnímu VFK souboru. 

Zdrojové kódy zásuvného modulu jsou šířeny pod licencí GNU GPL\footnote{https://raw.githubusercontent.com/ctu-osgeorel/qgis-vfk-plugin/master/LICENSE}. Zásuvný modul je možné stáhnout z oficiálního git repositáře\footnote{https://github.com/ctu-osgeorel/qgis-vfk-plugin-cpp}. \cite{cvut_vfkPlugin}

\begin{figure}[htb]
\centering
\includegraphics[width=\textwidth]{images/vfkPlugin-puvodni_okno.png}
\caption[VFK Plugin -- C++ verze]{VFK Plugin -- C++ verze (zdroj: \cite{cvut_vfkPlugin})}
\end{figure}

\newpage
\section{Funkcionalita}
Cílem zásuvného modulu je zjednodušit práci s daty katastru nemovitostí. Plugin byl proto navržen tak, aby byl schopen jednoduše a rychle řešit základní úlohy nad těmito daty. Mezi základní funkcionalitu patří možnost vyhledávání podle:

\begin{itemize}
 \item vlastníků,
 \item parcel,
 \item budov,
 \item jednotek.
\end{itemize}

\begin{figure}[htb]
    \centering
    \begin{subfigure}[b]{0.4\textwidth}
        \centering
        \includegraphics[width=\textwidth]{images/vfkPlugin-vlastnici.png}
        \caption{Vyhledávání vlastníků}
    \end{subfigure}
    ~
    \begin{subfigure}[b]{0.4\textwidth}
        \centering
        \includegraphics[width=\textwidth]{images/vfkPlugin-parcely.png}
        \caption{Vyhledávání parcel}
    \end{subfigure}
    \caption{Vyhledávací formuláře 1/2}
    \label{l_vyhledavani_1}
\end{figure}

\begin{figure}[htb]
    \centering
    \begin{subfigure}[b]{0.393\textwidth}
        \centering
        \includegraphics[width=\textwidth]{images/vfkPlugin-budovy.png}
        \caption{Vyhledávání budov}
    \end{subfigure}
    ~
    \begin{subfigure}[b]{0.4\textwidth}
        \centering
        \includegraphics[width=\textwidth]{images/vfkPlugin-jednotky.png}
        \caption{Vyhledávání jednotek}
    \end{subfigure}
    \caption{Vyhledávací formuláře 2/2}
    \label{l_vyhledavani_1}
\end{figure}

Ve vedlejším okně, prohlížeči dat, se zobrazují výsledky vyhledávání. Tato data lze interaktivně procházet, podobně jako je tomu ve webovém prohlížeči, jelikož jsou ukládána ve formě HTML stránek. Jsou zde implementovány základní vlastnosti HTML. Pro navigaci mezi jednotlivými prvky jsou používány hypertextové odkazy. Obsah okna prohlížeče je ukládán do historie, což usnadňuje uživateli vyhledávání stále se opakujících informací (není potřeba opakovaně provádět stejný databázový dotaz). Historii lze procházet pomocí tlačítek \texttt{Vpřed} a \texttt{Zpět}.

Pro zjištění aktuálního stavu o dané nemovitosti je zde implementováno propojení s aplikací \textbf{Nahlížení do katastru nemovitostí}\footnote{http://nahlizenidokn.cuzk.cz/}.

Veškeré výpisy zobrazené v prohlížeči zásuvného modulu lze exportovat, a to do následujících formátů:

\begin{itemize}
 \item \textbf{HTML}: umožňuje následné zobrazení ve webovém prohlížeči, případně import do textového procesoru,
 \item \textbf{zdrojový kód $\LaTeX$}: umožňuje vytvoření PDF nebo PS.
\end{itemize}

Zásuvný modul a mapové okno QGIS jsou navzájem propojené. Je tedy možné v mapovém okně zobrazit polohu vyhledaných prvků a opačně vyhledat informace o prvcích, které byly vybrány pomocí nástroje výběru QGIS právě v mapovém okně.

Uživatelská přívětivost používání zásuvného modulu je zvýšena \uv{dokovatelností} samotného okna pluginu. Modul je primárně ukotven k horní liště okna QGIS, ale je možné ho ukotvit i k ostatním okrajům. Okno modulu lze samozřejmě používat i samostatně (neukotvené).

Pro vrstvy parcel (\texttt{PAR}) a budov (\texttt{BUD}) byl vytvořen předdefinovaný vzhled ve formátu \textit{.xml}, respektive \textit{.qml}, se kterým QGIS nativně pracuje. Ukázka použité symbologie pro vrstvu parcel je uvedena v příloze \ref{l_priloha_symbologie}. 

Součástí zásuvného modulu je také stručná nápověda, která je po spuštění pluginu zobrazena stejně jako veškeré hledané informace v prohlížeči dat ve formátu HTML. Nápověda může být vyvolána i po kliknutí na příslušné tlačítko v liště panelu nástrojů VFK Pluginu. \cite{cvut_vfkPlugin}

\newpage
\section{Použití pluginu}

\subsection{Instalace a spuštění}
Stávající zásuvný modul, který je napsán v jazyce C++, je možné pod operačním systémem Ubuntu nainstalovat dle skriptů dostupných na stránkách pluginu\footnote{http://freegis.fsv.cvut.cz/gwiki/VFK\_/\_QGIS\_plugin}. Pro spuštění pod operačním systémem Windows, je potřeba stáhnout předkompilovanou verzi pro danou verzi QGIS a tu překopírovat do požadovaného adresáře. Podrobný návod je uveden opět na stránkách zásuvného modulu. 

Takto nainstalovaný modul lze potom vyhledat v QGIS v nabídce Pluginy $\rightarrow$ Spravovat zásuvné moduly. Zde lze plugin dohledat po zadání filtru \textit{VFK}. Spuštění pluginu je poté možné z hlavní nabídky QGIS Pluginy $\rightarrow$ VFK Plugin $\rightarrow$ VFK Plugin, nebo pomocí ikonky přímo z nástrojové lišty.

\subsection{Rozložení}
Okno zásuvného modulu se skládá ze tří hlavních částí. Vlevo se nachází hlavní panel nástrojů pro přepínání oken pro import souborů VFK a pro vyhledávání. Naprostou většinu pravé části tvoří prohlížeč dat, ve kterém se zobrazují výsledky vyhledávání a nápověda. Nad ním je poté nástrojová lišta, která obsahuje nástroje pro práci s prohlížečem, interakcí s mapou a nástroje pro export. 

\subsection{Načítání dat}
Pomocí zásuvného modulu je možné nahrát soubor VFK. Tento import je k~dispozici pomocí tlačítka \textit{Procházet} v nabídce \textit{Import VFK}. V tomto okně je dále možné zvolit, které vrstvy se budou importovat (PAR, BUD). V případě, kdy není zatrženo ani jedno pole, tak se načtou pouze popisná data. Bude tedy dostupné pouze vyhledávání bez interakce s mapovým oknem. 

\subsection{Vyhledávání}
Po úspěšném načtení dat je zpřístupněno vyhledávání. Veškeré informace vyhledané o~vlastnících, parcelách, budovách nebo jednotkách lze pohodlně procházet pomocí hypertextových odkazů v prohlížeči dat. Náhled vyhledaných informací je uveden na obrázku č. \ref{l_informace_o_parcele}.

\begin{figure}[htb]
\centering
\includegraphics[width=0.4\textwidth]{images/vfkPlugin-informace_o_parcele.png}
\caption[Ukázka vyhledaných informací k dané parcele]{Ukázka vyhledaných informací k~dané parcele}
\label{l_informace_o_parcele}
\end{figure}



% ==================================================================================
\clearpage
\chapter{Rozšíření stávající funkcionality}
V této kapitole bych se chtěl postupně věnovat jednotlivým krokům, které vedly k~rozšíření stávajícího zásuvného modulu programu QGIS o~podporu zpracování změnových vět souborů VFK. Spolu s tímto byly do pluginu přidány další funkcionality, jež jsou zmíněny na následujících řádcích kapitoly. 

\section{Usnadnění distribuce pluginu}
Jak už bylo zmíněno v kapitolách uvedených výše, stávající zásuvný modul byl napsán v jazyce C++. Což znamená, že před prvním použitím se musí plugin překompilovat. 

Kompilace pod operačním systémem Linux (Ubuntu) může být provedena pomocí skriptů\footnote{Návod dostupný na: http://freegis.fsv.cvut.cz/gwiki/VFK\_/\_QGIS\_plugin}, které vytvořili autoři původní verze modulu. 

Pod operačním systémem Microsoft Windows je situace poněkud komplikovanější. Pro každou verzi programu QGIS musí být plugin zkompilován samostatně vzhledem k typu operačního systému (32bit/64bit). Předkompilované verze mohou být staženy opět na stránkách pluginu, avšak nejsou zde uvedeny veškeré verze QGIS jak pro 32bit systém, tak pro 64bit systém. Uživatel jiné verzi QGIS (jiného typu operačního systému) by si tedy musel plugin zkompilovat sám, což může být pro nezkušené uživatele značná překážka. 

Výše zmíněné problémy značně snižují jak možnost většího rozšíření zásuvného modulu, tak komfort při jeho instalaci (aktualizaci). Řešením je přepis modulu do jazyka Python, čímž se plugin stane snadněji dostupným pro všechny uživatele bez rozdílu operačního systému, či verze programu QGIS.

\subsection{Přepis do jazyka Python}
Přepis zásuvného modulu do jazyka Python byl zásadním bodem této práce. Při přepisu jsem se snažil kód vytvářet co možná nejvíce podobný původnímu kódu v jazyku C++, a to hlavně z důvodu, aby byly zachovány veškeré funkcionality a vlastnosti tak, jak jsou známé stávajícím uživatelům nástroje.

Základní struktura kódu zásuvného modulu byla vygenerována pomocí doporučovaného nástroje \textbf{Plugin Builder}\footnote{http://g-sherman.github.io/Qgis-Plugin-Builder/} dostupného přímo z QGIS. Přepis byl značně ulehčen díky použití stejné knihovny Qt (respektive PyQt verze 4). V následujících ukázkách je porovnán původní kód v jazyce C++ (ukázka č. \ref{l_loadLayerC++}) s mnou napsaným kódem v jazyce Python (ukázka č. \ref{l_loadLayerPython}). Oba kódy slouží pro přidání vektorové vrstvy ze souboru VFK do okna QGIS.

\begin{lstlisting}[language=C++, 
		    caption=Kód pro načtení vektorové vrstvy v C++, 
		    keywordstyle=\color{blue}\ttfamily,
		    stringstyle=\color{red}\ttfamily,
		    commentstyle=\color{green}\ttfamily, morekeywords={QgsDebugMsg,QString,QgsVectorLayer,QList,QgsMapLayerRegistry},
		    label=l_loadLayerC++]
 void VfkMainWindow::loadVfkLayer( QString vfkLayerName )
{
  QgsDebugMsg( QString( "Loading vfk layer %1" ).arg( vfkLayerName ) );
  if ( mLoadedLayers.contains( vfkLayerName ) )
  {
    QgsDebugMsg( QString( "Vfk layer %1 is already loaded" ).arg( vfkLayerName ) );
    return;
  }
  QString composedURI = mLastVfkFile + "|layername=" + vfkLayerName;
  QgsVectorLayer *layer = new QgsVectorLayer( composedURI, vfkLayerName, "ogr" );
  mLoadedLayers.insert( vfkLayerName, layer->id() );
  setSymbology( layer );

  QList<QgsMapLayer *> myList;
  myList << layer;
  QgsMapLayerRegistry::instance()->addMapLayers( myList );
}
\end{lstlisting}

\newpage
\begin{lstlisting}[language=Python, 
		    caption=Kód pro načtení vektorové vrstvy v jazyce Python, 
		    keywordstyle=\color{blue}\ttfamily,
		    stringstyle=\color{red}\ttfamily,
		    commentstyle=\color{green}\ttfamily, morekeywords={qDebug,QString,QgsVectorLayer,QgsMapLayerRegistry,QMessageBox,self},
		    label=l_loadLayerPython]
def __loadVfkLayer(self, vfkLayerName):
"""
Method loads VFK layer.
:type vfkLayerName: str
"""
  qDebug("\n(VFK) Loading vfk layer {}".format(vfkLayerName))
  if vfkLayerName in self.__mLoadedLayers:
      qDebug(
	  "\n(VFK) Vfk layer {} is already loaded".format(vfkLayerName))
      return

  composedURI = self.__mDataSourceName + "|layername=" + vfkLayerName
  layer = QgsVectorLayer(composedURI, vfkLayerName, "ogr")
  if not layer.isValid():
      qDebug("\n(VFK) Layer failed to load!")

  self.__mLoadedLayers[vfkLayerName] = layer.id()

  try:
      self.__setSymbology(layer)
  except VFKWarning as e:
      QMessageBox.information(self, 'Load Style', e, QMessageBox.Ok)

  QgsMapLayerRegistry.instance().addMapLayer(layer)
\end{lstlisting}

\subsection{Instalace zásuvného modulu v QGIS}
Díky přepisu zásuvného modulu do jazyka Python je plugin v QGIS dostupný standardním způsobem, jako ostatní pluginy napsány v tomto jazyce. Plugin v současné době není dostupný v oficiálním repositáři QGIS. Pro instalaci pluginu je potřeba do QGIS přidat nový repositář spravovaný organizací OSGeoREL\footnote{http://geomatics.fsv.cvut.cz/research/osgeorel/}, pod kterou je tento plugin vyvíjen. Repositář je dostupný na adrese \url{http://geo.fsv.cvut.cz/osgeorel/qgis-plugins.xml}, viz obr. \ref{l_repo_detail}. Modul je šířený jako experimentální, proto musí být tato volba zohledněna při jeho instalaci, viz obr. \ref{l_qgis_plugins}. Okno pro zadání repositáře vyvoláme v záložce \textit{Plugins} $\rightarrow$ \textit{Manage and install plugins}.

\begin{figure}[htb]
\centering
\includegraphics[width=1\textwidth]{images/qgis_repo_plugins.png}
\caption[Přidání repositáře QGIS]{Přidání repositáře QGIS}
\label{l_qgis_plugins}
\end{figure}

\begin{figure}[htb]
\centering
\includegraphics[width=0.5\textwidth]{images/qgis_repo_detail.png}
\caption[Přidání repositáře QGIS -- detail]{Přidání repositáře QGIS -- detail}
\label{l_repo_detail}
\end{figure}

Po přidání repositáře a jeho aktualizaci je zásuvný modul dohledatelný standardním způsobem pod názvem \textbf{VFK Plugin}, viz obr. \ref{l_qgis_hledani_pluginu}.

\begin{figure}[htb]
\centering
\includegraphics[width=1\textwidth]{images/qgis_plugin_vyhledani.png}
\caption[Vyhledání zásuvného modulu]{Vyhledání zásuvného modulu}
\label{l_qgis_hledani_pluginu}
\end{figure}

Po správné instalaci zásuvného modulu se do nástrojové lišty přidá jeho ikonka (viz obr. \ref{l_plugin_ikona}). Plugin je poté dostupný po kliknutí na tuto ikonku nebo z menu \textit{Plugins} $\rightarrow$ \textit{VFK} $\rightarrow$ \textit{Otevřít prohlížeč VFK}.

\begin{figure}[H]
\centering
\includegraphics[scale=0.9]{images/vfkPluginIcon.png}
\caption[VFK Plugin -- ikona]{VFK Plugin -- ikona}
\label{l_plugin_ikona}
\end{figure}

% -----------------------------------------------------------------------
\clearpage
\section{Úprava GDAL VFK driveru}



\clearpage
% -----------------------------------------------------------------------
\section{Zpracování změnových vět}
\label{l_zpracovani_zmen}


\clearpage
\section{Popis tříd zásuvného modulu}

\subsubsection{ApplyChanges}
Třída, která se stará o aplikaci změn, které jsou pomocí \texttt{VFK Driveru GDAL} uloženy do databáze SQLite, na SQLite databázi vytvořenou ze stavového VFK souboru. Podrobnější popis zpracování změnových souborů touto třídou je k dispozici v kapitole \ref{l_zpracovani_zmen}.

\subsubsection{BudovySearchForm}
Jedná se o třídu, která se stará o vyhledávací prvky ve formuláři pro vyhledávání budov. Třída dědí od třídy \texttt{QWidget}. Vzhled uživatelského rozhraní této třídy byl navržen samostatně pomocí programu \texttt{Qt Designer}. Kód návrhu je uložen v souboru \texttt{ui\_budovysearchform.ui}, respektive v překompilované verzi pro jazyk Python \texttt{ui\_budovysearchform.py}. Ukázka konstruktoru třídy je uvedena v příloze \ref{l_budovySearchForm_konstruktor}.

\subsubsection{DocumentBuilder}
V třídě \texttt{DocumentBuilder} se sestavuje dokument, který bude zobrazen v prohlížeči dat (nebo exportován do některého z podporovaných formátů), na základě výsledků vyhledávání. O správné sestavení se stará metoda \texttt{buildHtml}, viz příloha \ref{l_buildHtml}. 

Třída mimo jiné obsahuje několik metod pro správné sestavení tabulek (např. \texttt{tableParcely(self, model, parIds, LVColumn)}), metody pro tvorbu jednotlivých částí listu vlastnictví (např. \texttt{partTelesoB1(self, parIds, budIds, jedIds, opsubIds, forLV)}), nebo metody pro zobrazení detailu o vyhledávaném tělese (např. \texttt{pageBudova(self, id)}).

Důležitou metodou v této třídě je metoda \texttt{pageHelp(self)}, díky které je možné zobrazit krátkou nápovědu k zásuvnému modulu, a metoda \texttt{saveDefinitionPoint( self, id, nemovitost)}, která slouží k získání a uložení definičního bodu vyhledané nemovitosti.

\subsubsection{Domains}
V této třídě jsou jednoduché metody pro zjištění domény prvku na základě vstupního argumentu. Je zde metoda \texttt{anoNe(an)}, která testuje, zda je vstupní argument rovný \uv{a}, a podle toho vrací \texttt{True/False}. Metoda \texttt{cpCe(kod)} převádí hodnotu vstupního argumentu na řetězec \uv{Číslo popisné} / \uv{Číslo evidenční}. Metoda \texttt{druhUcastnika(kod)} kontroluje, o jaký typ subjektu se jedná (např. právnická osoba). Poslední metodou je metoda \texttt{rodinnyStav(kod)}, která vrací rodinný stav subjektu na základě zadaného vstupního kódu.

\subsubsection{HtmlDocument}
Třída \texttt{HtmlDocument} je potomkem třídy \texttt{VfkDocument}(viz \ref{l_vfkDocument}). Jsou v ní obsaženy metody pro tvorbu jednotlivých částí kódu v jazyce \textit{HTML}. Vhodnou kombinací těchto částí je poté sestaven celý \textit{HTML} dokument. Takto připravený dokument je poté možné vyexportovat pomocí příslušné funkcionality a zobrazit v libovolném prohlížeči.

\subsubsection{JednotkySearchForm}
V této třídě se řeší vyhledávání jednotek pomocí vyhledávacích formulářů navržených v programu \texttt{Qt Designer}. Vyhledávat lze podle čísla jednotky, domovního čísla, čísla parcely, ke které jednotka patří, listu vlastnictví nebo způsobu využití jednotky. Třída je potomkem třídy \texttt{QWidget}.

\subsubsection{LatexDocument}
Třída \texttt{LatexDocument} obsahuje metody, pomocí nichž se vytváří části kódu, které jsou nakonec sestaveny v  $\LaTeX$ dokument. Takto sestavený dokument je poté vyexportován pomocí příslušné funkcionality modulu a je připravený pro export například od formátu \textit{PDF} (\texttt{pdflatex}). Třída dědí od rodičovské třídy \texttt{VfkDocument}. 

\subsubsection{MainApp}
Třída \texttt{MainApp} je stěžejní třídou zásuvného modulu. Dědí od tříd \texttt{QDockWidget}, \texttt{QMainWindow} a \texttt{Ui\_MainApp}. Obsahuje metody pro načítání VFK souborů, nastavuje symbologii načtených vrstev, propojuje vyhledávání s mapovým oknem QGIS, stará se o exporty do definovaných formátů. Ve třídě se také přidává funkcionalita jednotlivým prvkům zásuvného modulu.

\subsubsection{OpenThread}
Tato třída se stará o načítání jednotlivých \textit{VFK} souborů v separátním vlákně. Díky použití jiného vlákna pro načítání souborů, se práce se zásuvným modulem stává plynulejší. Třída dědí od třídy \texttt{QThread} dostupné v použité knihovně \texttt{PyQt4}. Obsahuje jedinou metodu (\texttt{run(self)}) pro spuštění separátního vlákna, jejíž předpis je dostupný v příloze č. \ref{l_thread_run}.

\subsubsection{ParcelySearchForm}
Třída \texttt{ParcelySearchForm} stejně jako předešlé třídy pro vyhledávání dědí od rodičovské třídy \texttt{QWidget}. Obsahuje metody, které pracují s navrženými grafickými prvky. Díky těmto metodám je možné vyhledávat parcely podle parcelního čísla, typu parcely, druhu pozemku nebo listu vlastnictví.

\subsubsection{RichTextDocument}
Třída se stará o tvorbu \textit{HTML} dokumentu, který obsahuje výsledky vyhledávání a je zobrazen v prohlížeči dat. Třída dědí od rodičovské třídy \texttt{VfkDocument}. 

\subsubsection{SearchFormController}
Třída \texttt{SearchFormController} získává hodnoty ze vstupních formulářů, které jsou obsahem tříd pro vyhledávání  (\texttt{BudovySearchForm}, \texttt{JednotkySearchForm}, \texttt{Par\-celySearchForm}, \texttt{VlasniciSearchForm}). Z těchto vstupních hodnot poté sestavuje vyhledávací dotaz, který pomocí signálu \texttt{actionTriggered} emituje dále. Ukázka kódu pro sestavení vyhledávacího dotazu pro parcely je uvedena v příloze \ref{l_search_parcely}.

\subsubsection{VfkDocument}
\label{l_vfkDocument}
Jedná se o abstraktní třídu, ze které jsou odvozeny třídy pro sestavení dokumentů v různých formátech (\texttt{HtmlDocument}, \texttt{LatexDocument}, \texttt{RichTextDocument}). Jsou zde implementovány základní metody, které zajišťují správné sestavení dokumentu.

\subsubsection{VfkPlugin}
Tato třída implementuje zásuvný modul do programu QGIS. Jsou zde uvedeny metody, které vytvářejí instanci pluginu v programu QGIS po jeho instalaci, nebo metody, které tuto instanci ruší. Tato třída byla vygenerována automaticky při tvorbě základního schéma pluginu nástrojem \textit{Plugin Builder} a později upravena dle konkrétních potřeb.

\subsubsection{VfkTableModel}
Tato třída dědí od třídy \texttt{QSqlQueryModel}, která poskytuje model pro čtení SQL dotazů. Třída komunikuje s připojenou databází, která je do třídy předávána pomocí argumentu \texttt{connectionName}. Ke komunikaci jsou používány standardní SQL dotazy, které jsou do databáze předávány metodou \texttt{self.setQuery(query, QSqlDatabase.database(self.\_\_mConnectionName))}.

\subsubsection{VfkTextBrowser}
\texttt{VfkTextBrowser} je třída, která dědí od třídy \texttt{QTextBrowser}. Třída byla stanovena jako \uv{zástupná} pro prvek \texttt{vfkBrowser} v okně zásuvného modulu. Tento prvek slouží jako prohlížeč dat.

\subsubsection{VlastniciSearchForm}
Jedná se o třídu, která slouží k vyhledávání vlastníků podle jména, typu (fyzická osoba, právnická osoba, společné jmění manželů), rodného čísla (případně IČO) nebo listu vlastnictví. Metody třídy získávají hodnoty vstupních polí / \uv{checkboxů}.


\clearpage
\chapter*{Závěr}

%======================CITACE=========================================
\clearpage
\rhead{{\rightmark}}	% vpravo název kapitoly
\renewcommand{\refname}{Použitá literatura}
\bibliography{citace}
\bibliographystyle{czechiso}

%======================SEZNAM OBRÁZKŮ===============================
\clearpage
\listoffigures

%======================SEZNAM TABULEK================================
\clearpage
\listoftables

%====================== PŘÍLOHY ========================================
\newpage
\appendix

\setcounter{page}{1}   	% nastaví čítač stránek znovu od jedné
\pagenumbering{Roman} % číslování arabskými

\chapter{Ukázka definice symbologie ve formátu \textit{.qml} pro vrstvu parcel}
\label{l_priloha_symbologie}

\begin{lstlisting}[language=XML]
<!DOCTYPE qgis PUBLIC 'http://mrcc.com/qgis.dtd' 'SYSTEM'>
<qgis version="1.9.90-Alpha" minimumScale="0" maximumScale="1e+08" hasScaleBasedVisibilityFlag="0">
  <transparencyLevelInt>255</transparencyLevelInt>
  <renderer-v2 symbollevels="0" type="singleSymbol">
    <symbols>
      <symbol outputUnit="MM" alpha="1" type="fill" name="0">
        <layer pass="0" class="SimpleFill" locked="0">
          <prop k="color" v="0,0,0,255"/>
          <prop k="color_border" v="0,0,0,255"/>
          <prop k="offset" v="0,0"/>
          <prop k="style" v="no"/>
          <prop k="style_border" v="solid"/>
          <prop k="width_border" v="0.26"/>
        </layer>
      </symbol>
    </symbols>
    <rotation field=""/>
    <sizescale field=""/>
  </renderer-v2>
  <customproperties>
    <property key="labeling" value="pal"/>
    <property key="labeling/addDirectionSymbol" value="false"/>
    <property key="labeling/bufferColorB" value="255"/>
    <property key="labeling/bufferColorG" value="255"/>
    <property key="labeling/bufferColorR" value="255"/>
    <property key="labeling/bufferSize" value="0.5"/>
    <property key="labeling/dataDefinedProperty0" value=""/>
    <property key="labeling/dataDefinedProperty1" value=""/>
    <property key="labeling/dataDefinedProperty10" value=""/>
    <property key="labeling/dataDefinedProperty11" value=""/>
    <property key="labeling/dataDefinedProperty12" value=""/>
    <property key="labeling/dataDefinedProperty13" value=""/>
    <property key="labeling/dataDefinedProperty14" value=""/>
    <property key="labeling/dataDefinedProperty2" value=""/>
    <property key="labeling/dataDefinedProperty3" value=""/>
    <property key="labeling/dataDefinedProperty4" value=""/>
    <property key="labeling/dataDefinedProperty5" value=""/>
    <property key="labeling/dataDefinedProperty6" value=""/>
    <property key="labeling/dataDefinedProperty7" value=""/>
    <property key="labeling/dataDefinedProperty8" value=""/>
    <property key="labeling/dataDefinedProperty9" value=""/>
    <property key="labeling/decimals" value="0"/>
    <property key="labeling/dist" value="0"/>
    <property key="labeling/distInMapUnits" value="false"/>
    <property key="labeling/enabled" value="true"/>
    <property key="labeling/fieldName" value="CASE WHEN (length(&quot;PODDELENI_CISLA_PAR&quot; ) = 0  OR &quot;PODDELENI_CISLA_PAR&quot; = '0') THEN  KMENOVE_CISLO_PAR ELSE &quot;KMENOVE_CISLO_PAR&quot;  || '/' || &quot;PODDELENI_CISLA_PAR&quot; END"/>
    <property key="labeling/fontFamily" value="DejaVu Sans"/>
    <property key="labeling/fontItalic" value="true"/>
    <property key="labeling/fontSize" value="9"/>
    <property key="labeling/fontSizeInMapUnits" value="false"/>
    <property key="labeling/fontStrikeout" value="false"/>
    <property key="labeling/fontUnderline" value="false"/>
    <property key="labeling/fontWeight" value="50"/>
    <property key="labeling/formatNumbers" value="false"/>
    <property key="labeling/isExpression" value="true"/>
    <property key="labeling/labelPerPart" value="false"/>
    <property key="labeling/mergeLines" value="false"/>
    <property key="labeling/minFeatureSize" value="0"/>
    <property key="labeling/obstacle" value="true"/>
    <property key="labeling/placement" value="1"/>
    <property key="labeling/placementFlags" value="0"/>
    <property key="labeling/plussign" value="true"/>
    <property key="labeling/priority" value="5"/>
    <property key="labeling/scaleMax" value="5000"/>
    <property key="labeling/scaleMin" value="1"/>
    <property key="labeling/textColorB" value="0"/>
    <property key="labeling/textColorG" value="0"/>
    <property key="labeling/textColorR" value="0"/>
    <property key="labeling/wrapChar" value=""/>
  </customproperties>
  <displayfield>ID</displayfield>
  <label>0</label>
  <labelattributes>
    <label fieldname="" text="Label"/>
    <family fieldname="" name="Sans"/>
    <size fieldname="" units="pt" value="12"/>
    <bold fieldname="" on="0"/>
    <italic fieldname="" on="0"/>
    <underline fieldname="" on="0"/>
    <strikeout fieldname="" on="0"/>
    <color fieldname="" red="0" blue="0" green="0"/>
    <x fieldname=""/>
    <y fieldname=""/>
    <offset x="0" y="0" units="pt" yfieldname="" xfieldname=""/>
    <angle fieldname="" value="0" auto="0"/>
    <alignment fieldname="" value="center"/>
    <buffercolor fieldname="" red="255" blue="255" green="255"/>
    <buffersize fieldname="" units="pt" value="1"/>
    <bufferenabled fieldname="" on=""/>
    <multilineenabled fieldname="" on=""/>
    <selectedonly on=""/>
  </labelattributes>
  <edittypes>
    <edittype type="0" name="BUD_ID"/>
    <edittype type="0" name="CENA_NEMOVITOSTI"/>
    <edittype type="0" name="DATUM_VZNIKU"/>
    <edittype type="0" name="DATUM_ZANIKU"/>
    <edittype type="0" name="DEFINICNI_BOD_PAR"/>
    <edittype type="0" name="DIL_PARCELY"/>
    <edittype type="0" name="DRUH_CISLOVANI_PAR"/>
    <edittype type="0" name="DRUPOZ_KOD"/>
    <edittype type="0" name="ID"/>
    <edittype type="0" name="IDENT_BUD"/>
    <edittype type="0" name="KATUZE_KOD"/>
    <edittype type="0" name="KATUZE_KOD_PUV"/>
    <edittype type="0" name="KMENOVE_CISLO_PAR"/>
    <edittype type="0" name="MAPLIS_KOD"/>
    <edittype type="0" name="PAR_ID"/>
    <edittype type="0" name="PAR_TYPE"/>
    <edittype type="0" name="PKN_ID"/>
    <edittype type="0" name="PODDELENI_CISLA_PAR"/>
    <edittype type="0" name="PRIZNAK_KONTEXTU"/>
    <edittype type="0" name="RIZENI_ID_VZNIKU"/>
    <edittype type="0" name="RIZENI_ID_ZANIKU"/>
    <edittype type="0" name="STAV_DAT"/>
    <edittype type="0" name="TEL_ID"/>
    <edittype type="0" name="TYP_PARCELY"/>
    <edittype type="0" name="VYMERA_PARCELY"/>
    <edittype type="0" name="ZDPAZE_KOD"/>
    <edittype type="0" name="ZPURVY_KOD"/>
    <edittype type="0" name="ZPVYPA_KOD"/>
  </edittypes>
  <editform></editform>
  <editforminit></editforminit>
  <annotationform></annotationform>
  <attributeactions/>
  <overlay display="false" type="diagram">
    <renderer item_interpretation="linear">
      <diagramitem size="0" value="0"/>
      <diagramitem size="0" value="0"/>
    </renderer>
    <factory sizeUnits="MM" type="Pie">
      <wellknownname>Pie</wellknownname>
      <classificationfield>0</classificationfield>
    </factory>
    <scalingAttribute>0</scalingAttribute>
  </overlay>
</qgis>
\end{lstlisting}

\chapter{Ukázky kódu}

\section{Konstruktor třídy BudovySearchForm}

\begin{lstlisting}[language=Python, 
		    keywordstyle=\color{blue}\ttfamily,
		    stringstyle=\color{red}\ttfamily,
		    commentstyle=\color{green}\ttfamily, morekeywords={qDebug,QString,QgsVectorLayer,QgsMapLayerRegistry,QMessageBox,self},
		    label=l_budovySearchForm_konstruktor]
class BudovySearchForm(QWidget):

    def __init__(self, parent=None):
        super(BudovySearchForm, self).__init__(parent)

        # Set up the user interface from Designer.
        self.ui = Ui_BudovySearchForm()
        self.ui.setupUi(self)

        self.__mZpusobVyuzitiModel = QAbstractItemModel
\end{lstlisting}


\section{Metoda buildHtml}

\begin{lstlisting}[language=Python, 
		    keywordstyle=\color{blue}\ttfamily,
		    stringstyle=\color{red}\ttfamily,
		    commentstyle=\color{green}\ttfamily, morekeywords={qDebug,QString,QgsVectorLayer,QgsMapLayerRegistry,QMessageBox,self},
		    label=l_buildHtml]
def buildHtml(self, document, taskMap):
"""
:type document: VfkDocument
:type taskMap: dict
"""
        self.__mCurrentPageParIds = []
        self.__mCurrentPageBudIds = []
        self.__mCurrentDefinitionPoint.first = ''
        self.__mCurrentDefinitionPoint.second = ''

        self.__mDocument = document
        self.__mDocument.header()

        if taskMap["page"] == "help":
            self.pageHelp()

        if self.__mHasConnection:
            if taskMap["page"] == "tel":
                self.pageTeleso(taskMap["id"])
            elif taskMap["page"] == "par":
                self.pageParcela(taskMap["id"])
            elif taskMap["page"] == "bud":
                self.pageBudova(taskMap["id"])
            elif taskMap["page"] == "jed":
                self.pageJednotka(taskMap["id"])
            elif taskMap["page"] == "opsub":
                self.pageOpravnenySubjekt(taskMap["id"])
            elif taskMap["page"] == "seznam":
                if taskMap["type"] == "id":
                    if "parcely" in taskMap:
                        self.pageSeznamParcel(taskMap["parcely"].split(","))
                    if "budovy" in taskMap:
                        self.pageSeznamBudov(taskMap["budovy"].split(","))
                elif taskMap["type"] == "string":
                    if "opsub" in taskMap:
                        self.pageSeznamOsob(taskMap['opsub'].split(","))
            elif taskMap["page"] == "search":
                if taskMap["type"] == "vlastnici":
                    self.pageSearchVlastnici(
                        taskMap["jmeno"], taskMap["rcIco"],
                                             taskMap["sjm"], taskMap["opo"],
                                             taskMap["ofo"], taskMap["lv"])
                elif taskMap["type"] == "parcely":
                    self.pageSearchParcely(
                        taskMap["parcelniCislo"], taskMap["typ"], taskMap["druh"], taskMap["lv"])
                elif taskMap["type"] == "budovy":
                    self.pageSearchBudovy(
                        taskMap["domovniCislo"], taskMap[
                            "naParcele"], taskMap["zpusobVyuziti"],
                                          taskMap["lv"])
                elif taskMap["type"] == "jednotky":
                    self.pageSearchJednotky(
                        taskMap["cisloJednotky"], taskMap[
                            "domovniCislo"], taskMap["naParcele"],
                                            taskMap["zpusobVyuziti"], taskMap["lv"])
        self.__mDocument.footer()
        return
\end{lstlisting}

\section{Načtení souborů v separátním vlákně}
\begin{lstlisting}[language=Python, 
		    keywordstyle=\color{blue}\ttfamily,
		    stringstyle=\color{red}\ttfamily,
		    commentstyle=\color{green}\ttfamily, morekeywords={qDebug,QString,QgsVectorLayer,QgsMapLayerRegistry,QMessageBox,self},
		    label=l_thread_run]
 def run(self):
    # load all VFK files
    for i, vfkFile in enumerate(self.vfk_files):
	self.working.emit(vfkFile)
	self.nextLayer = True

	while self.nextLayer:
	    self.sleep(1)
\end{lstlisting}

\section{Vyhledání parcel -- ukázka URL}
\begin{lstlisting}[language=Python, 
		    keywordstyle=\color{blue}\ttfamily,
		    stringstyle=\color{red}\ttfamily,
		    commentstyle=\color{green}\ttfamily, morekeywords={QUrl,QString,self},
		    label=l_search_parcely]
def __searchParcely(self):
"""
Method creates and emits query for searching of plots.
"""
    parcelniCislo = self.__forms.parcely.parcelniCislo()
    typ = int(self.__forms.parcely.typParcely())
    druh = self.__forms.parcely.druhPozemkuKod()
    lv = self.__forms.parcely.lv()

    # build url
    url = QUrl(u"showText?page=search&type=parcely&parcelniCislo={}&typ={}&druh={}&lv={}"
		.format(parcelniCislo, typ, druh, lv))
    self.actionTriggered.emit(url)
\end{lstlisting}




%-----------------------------------------------


\end{document}