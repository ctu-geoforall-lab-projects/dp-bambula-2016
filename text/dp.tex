%----------------------------------------------------------------------------------------
%	PACKAGES AND OTHER DOCUMENT CONFIGURATIONS
%----------------------------------------------------------------------------------------

\documentclass[12pt]{article}

\usepackage[utf8]{inputenc}	% kódování textu
\usepackage[czech]{babel}		% zavedení češtiny
\usepackage[IL2]{fontenc}
\usepackage[pdftex]{hyperref}	% veškeré klikací odkazy
\usepackage{a4wide}
\usepackage{indentfirst}	% odsazení prvního řádku odstavce
\usepackage{amsmath,amsfonts,amssymb}	% matematika
\usepackage{graphicx}	% grafika
\usepackage{multirow}	% slouceni radek v tabulce
\usepackage{multicol}	% slouceni sloupcu v tabulce
\usepackage{longtable}	% rozdeleni tabulky pres vice stran
\usepackage{enumerate}	% seznamy
\usepackage{float}
\usepackage{booktabs}	% professional tables
\usepackage{lscape}		% stranka na sirku
\usepackage{fancyhdr}
\usepackage{url}
\usepackage{array}
\usepackage{subfigure}


% nastavení příkazu URL podle normy ISO
%\DeclareUrlCommand\url{\def\UrlLeft{< }\def\UrlRight{ >} \urlstyle{tt}}	% správné zobrazení www

\setlength{\parskip}{1ex}


\begin{document}
% ===============================================================
% NASTAVENI ZAHLAVI A ZAPATI	
% ===============================================================
% uvodni stranka zahlavi ani zapati mit nebude
\thispagestyle{empty}
 % vlevo text a název aktuální sekce
\lhead{\includegraphics[scale=0.2]{images/cvut_logo.png}\hspace{10pt}ČVUT v~Praze}
 % vpravo název kapitoly
\rhead{{\rightmark}}

 % nastavime pouziti naseho stylu
\pagestyle{fancy}

\renewcommand{\sectionmark}[1]{\markright{#1}{}}
\renewcommand{\subsectionmark}[1]{\markright{#1}{}}

\fancypagestyle{plain}{
  \fancyhead{} % na prázdných stránkách nechci záhlaví
  \renewcommand{\headrulewidth}{0pt} % ani linku
}

% =========================
% TITULNÍ STRANA 11111111111
% =========================

% \begin{titlepage}
% 
% \newcommand{\HRule}{\rule{\linewidth}{0.5mm}} % Defines a new command for the horizontal lines, change thickness here
% 
% \center % Center everything on the page
%  
% %----------------------------------------------------------------------------------------
% %	HEADING SECTIONS
% %----------------------------------------------------------------------------------------
% 
% \textsc{\LARGE České vysoké učení technické v~Praze}\\[0.5cm] % Name of your university/college
% \textsc{\Large Fakulta stavební} % Major heading such as course name
% 
% 
% %----------------------------------------------------------------------------------------
% %	TITLE SECTION
% %----------------------------------------------------------------------------------------
% 
% \vfill
% 
% \textsc{\LARGE Diplomová práce}
%  
% %----------------------------------------------------------------------------------------
% %	AUTHOR SECTION
% %----------------------------------------------------------------------------------------
% \vfill
% \begin{minipage}{0.4\textwidth}
% \begin{flushleft} 
% \large 2016	% datum
% \end{flushleft}
% \end{minipage}
% ~
% \begin{minipage}{0.4\textwidth}
% \begin{flushright} 
% \large Bc. Štěpán \textsc{Bambula} \\ % Your name
% \end{flushright}
% \end{minipage}\\[1cm]
%  
% \end{titlepage}
% % -------------------------------------------------------------------------------------------------------------


% =========================
% TITULNÍ STRANA 2222222222
% =========================

\begin{titlepage}
\center % Center everything on the page
 
%----------------------------------------------------------------------------------------
%	HEADING SECTIONS
%----------------------------------------------------------------------------------------

\textsc{\LARGE České vysoké učení technické v~Praze}\\[0.5cm] % Name of your university/college
\textsc{\Large Fakulta stavební}\\[0.5cm] % Major heading such as course name
\textsc{\large Studijní program: Geodézie a kartografie}\\[0.5cm] % Minor heading such as course title
\textsc{\large Studijní obor: Geomatika}\\[0.5cm] % Minor heading such as course title

%----------------------------------------------------------------------------------------
%	LOGO SECTION
%----------------------------------------------------------------------------------------

\vspace{50pt}

\includegraphics[scale=1.2]{images/cvut_logo.png}  % Include a department/university logo 

%----------------------------------------------------------------------------------------
%	TITLE SECTION
%----------------------------------------------------------------------------------------

\vspace{40pt}

\textsc{\Large Diplomová práce}\\
\vfill
\textsc{\LARGE  Rozšíření nástroje pro práci s~katastrálními daty v~programu QGIS}\\[0.5cm] % Title of your document

\textsc{\Large QGIS VFK Plugin Improvements}
 
 
 %----------------------------------------------------------------------------------------
%	VEDOUCÍ PRÁCE
%----------------------------------------------------------------------------------------
 \vfill
\large Vedoucí práce: Ing. Martin \textsc{Landa}, Ph.D.

Katedra geomatiky
 
%----------------------------------------------------------------------------------------
%	AUTHOR SECTION
%----------------------------------------------------------------------------------------
\vfill
\begin{minipage}{0.4\textwidth}
\begin{flushleft} 
\large 2016	% datum
\end{flushleft}
\end{minipage}
~
\begin{minipage}{0.4\textwidth}
\begin{flushright} 
\large Bc. Štěpán \textsc{Bambula} \\ % Your name
\end{flushright}
\end{minipage}\\[1cm]
 
\end{titlepage}

%========================= ZADÁNÍ PRÁCE =================================
\clearpage
\pagestyle{empty}

\vspace*{\fill}
\begin{center}
\textsc{\Large Zde vložit zadání práce!!!}
\end{center}

\vspace*{\fill}

%========================= ABSTRAKT =====================================
\clearpage

\hfill

\noindent
\textsc{\Large Abstrakt}

\vspace{12pt}

Cílem diplomové práce je rozšířit projekt laboratoře OSGeoREL ČVUT v~Praze zaměřený na práci s~katastrálními daty poskytovanými ve výměnném formátu VFK v~prostředí open source nástroje QGIS. Práce navazuje na již existující nástroj implementovaný jako tzv. zásuvný modul a rozšiřuje ho o~novou funkctionalitu a to především zpracování a vizualizaci datových vět změnových souborů VFK. Druhotným cílem je usnadnění distribuce zásuvného modulu v~prostředí QGIS s~důrazem na jeho přenositelnost.

\vspace{32pt}

\noindent
\textsc{\Large Klíčová slova}

\vspace{12pt}

VFK, QGIS, ČUZK, Python, C++, PyQt, GDAL


\vfill

\noindent
\textsc{\Large Abstract}

\vspace{12pt}


\vspace{32pt}

\noindent
\textsc{\Large Keywords}

\vspace{12pt}
VFK, QGIS, CUZK, Python, C++, PyQt, GDAL


\vfill

%========================= PROHLÁŠENÍ ==================================
\clearpage
\vspace*{\fill}

\noindent
\textsc{\Large Prohlášení}

\vspace{12pt}
Prohlašuji, že jsem diplomovou práci na téma \uv{Rozšíření nástroje pro práci s~katastrálními daty v~programu QGIS} vypracoval samostatně. Všechny podklady, ze kterých jsem čerpal, jsou uvedeny v~seznamu použité literatury.

\vspace{24pt}
\noindent
\begin{minipage}{0.4\textwidth}
\begin{flushleft}
\center 
V~Praze dne \dots \dots \dots \\
\end{flushleft}
\end{minipage}
~
\begin{minipage}{0.8\textwidth}
\begin{flushright} 
\vspace{20pt}
\center
\dots \dots \dots \dots \dots \dots \dots \dots \\
Štěpán Bambula
\end{flushright}
\end{minipage}\\[2cm]


%========================= PODĚKOVÁNÍ ==================================
\clearpage
\vspace*{\fill}

\noindent
\textsc{\Large Poděkování}

\vspace{12pt}

\vspace{2cm}

%======================POUZITE ZKRATKY===============================
\clearpage
\rhead{SEZNAM POUŽITÝCH ZKRATEK}		 % vpravo název kapitoly
\section*{Seznam použitých zkratek}

\begin{description}
\item[VFK] Výměnný formát katastru nemovitostí
\item[ČUZK] Český úřad zeměměřický a katastrální
\item[GDAL] Geospatial Data Abstraction Library
\end{description}


%=========================OBSAH=======================================
\clearpage
\rhead{{\rightmark}}		% vpravo název kapitoly
\tableofcontents

%============================ÚVOD====================================
\clearpage
\pagestyle{fancy}		% nastaví styl stránky a číslování
\setcounter{page}{1}   	% nastaví čítač stránek znovu od jedné
\pagenumbering{arabic} % číslování arabskými
\rhead{Úvod}		 % vpravo název kapitoly
\section*{Úvod}
\addcontentsline{toc}{section}{Úvod}




\clearpage
\rhead{{\rightmark}}
\section{Dostupné nástroje pro práci s~VFK}


\clearpage
\section{Použité technologie}

\subsection{QGIS}

\subsection{Python}

Jazyk Python je objektově orientovaný programovací jazyk, který efektivně používá víceúrovňové datové typy. Jedná se o~jazyk interpretovaný, čímž se jeví jako ideální nástroj pro psaní skriptů, ale i rychlý vývoj aplikací. Je vyvíjen jako open-source\footnote{Open-source software je takový software, k~němuž zákazník dostane od jeho tvůrce zdrojový kód a může jej dále upravovat. Jednotlivé definice termínu \uv{open source} se liší zvláště v~podmínkách pro další distribuci softwaru.\cite{abclinuxu_opensource}} software, díky čemuž se stává použitelným na velkém množství platforem (Linux, Windows, MacOS, \dots). Jazyk je rozšířitelný o~široké spektrum modulů, které umožňují řešit problematiku takřka z~jakékoli oblasti. V~současné době je Python vyvíjen ve dvou verzích, ve verzi 2.x a v~novější verzi 3.x.
\cite{dive_into_python}
\cite{python_web}

\subsection{PyQt}

PyQt je modul, který zpřístupňuje knihovnu Qt pro programovací jazyk Python. Spolu s~PySide se jedná o~nejznámější a nejpoužívanější modul pro Python postavený nad knihovnou Qt. Je vyvíjen britskou firmou Riverbank Computing ve dvou verzích. Ve verzi 4, podporující knihovnu Qt 4, a ve verzi 5, která podporuje novější verzi Qt knihovny. Modul je dostupný na všech platformách, které podporují knihovnu Qt (Windows, MacOS/X a~Linux). PyQt je šířeno pod tzv. dvojí licencí, GNU GPL v3 a  Riverbank Commercial License. Spolu s~těmito licencemi je dostupné i pod komerční licencí.

Pro grafický návrh aplikace je vhodné použít nativní grafické uživatelské rozhraní Qt Designer. Výstupem z~tohoto programu je soubor obsahující vzhled aplikace ve formátu \textit{.xml}. PyQt je poté schopné tento formát převést do kódu jazyka Python. Pro komunikaci mezi objekty je využíváno signálů a slotů, díky čemuž je vytvoření komponent velice snadné.

PyQt v~sobě kombinuje mocnost knihovny Qt s~jednoduchostí jazyka Python, což z~něj dělá výkonný nástroj pro vývoj grafických aplikací.
\cite{pyqt}

\clearpage
\section{Formát VFK}

\cite{dp_landa}
\cite{vfk_struktura}












%======================CITACE=========================================
\clearpage
\rhead{{\rightmark}}		% vpravo název kapitoly
\renewcommand{\refname}{Použitá literatura}
\bibliography{citace}
\bibliographystyle{czechiso}

%======================SEZNAM OBRÁZKŮ===============================
\clearpage
\listoffigures

%======================SEZNAM TABULEK================================
\clearpage
\listoftables

%====================== PŘÍLOHY ========================================
\newpage
\appendix

\setcounter{page}{1}   	% nastaví čítač stránek znovu od jedné
\pagenumbering{Roman} % číslování arabskými

%-----------------------------------------------


\end{document}