%----------------------------------------------------------------------------------------
%	PACKAGES AND OTHER DOCUMENT CONFIGURATIONS
%----------------------------------------------------------------------------------------

%\documentclass[12pt]{article}
\documentclass[a4paper,12pt,oneside]{book}

\usepackage[utf8]{inputenc}	% kódování textu
\usepackage[czech]{babel}		% zavedení češtiny
\usepackage[IL2]{fontenc}
\usepackage[pdftex]{hyperref}	% veškeré klikací odkazy
\usepackage{a4wide}
\usepackage{indentfirst}	% odsazení prvního řádku odstavce
\usepackage{amsmath,amsfonts,amssymb}	% matematika
\usepackage{graphicx}	% grafika
\usepackage{multirow}	% slouceni radek v tabulce
\usepackage{multicol}	% slouceni sloupcu v tabulce
\usepackage{longtable}	% rozdeleni tabulky pres vice stran
\usepackage{enumerate}	% seznamy
\usepackage{float}
\usepackage{booktabs}	% professional tables
\usepackage{lscape}		% stranka na sirku
\usepackage{fancyhdr}
\usepackage{url}
\usepackage{array}
\usepackage{subfigure}



\usepackage[%
%top=40mm,
%bottom=35mm,
%left=40mm,
%right=30mm
top=40mm,
bottom=35mm,
left=35mm,
right=25mm
]{geometry}

% nastavení příkazu URL podle normy ISO
%\DeclareUrlCommand\url{\def\UrlLeft{< }\def\UrlRight{ >} \urlstyle{tt}}	% správné zobrazení www

\setlength{\parskip}{1ex}


\begin{document}
% ===============================================================
% NASTAVENI ZAHLAVI A ZAPATI	
% ===============================================================
% uvodni stranka zahlavi ani zapati mit nebude
\thispagestyle{empty}
 % vlevo text a název aktuální sekce
\lhead{\includegraphics[scale=0.2]{images/cvut_logo.png}\hspace{10pt}ČVUT v~Praze}
 % vpravo název kapitoly
\rhead{{\rightmark}}

 % nastavime pouziti naseho stylu
\pagestyle{fancy}

\renewcommand{\chaptermark}[1]{\markright{#1}{}}
%\renewcommand{\sectionmark}[1]{\markright{#1}{}}

\fancypagestyle{plain}{
  \fancyhead{} % na prázdných stránkách nechci záhlaví
  \renewcommand{\headrulewidth}{0pt} % ani linku
}

% =========================
% TITULNÍ STRANA 11111111111
% =========================

\begin{titlepage}
 
\newcommand{\HRule}{\rule{\linewidth}{0.5mm}} % Defines a new command for the horizontal lines, change thickness here

\center % Center everything on the page

%----------------------------------------------------------------------------------------
%	HEADING SECTIONS
%----------------------------------------------------------------------------------------

\textsc{\LARGE České vysoké učení technické v~Praze}\\[0.5cm] % Name of your university/college
\textsc{\Large Fakulta stavební} % Major heading such as course name


%----------------------------------------------------------------------------------------
%	TITLE SECTION
%----------------------------------------------------------------------------------------

\vfill

\textsc{\LARGE Diplomová práce}

%----------------------------------------------------------------------------------------
%	AUTHOR SECTION
%----------------------------------------------------------------------------------------
\vfill
\begin{minipage}{0.4\textwidth}
\begin{flushleft} 
\large 2016	% datum
\end{flushleft}
\end{minipage}
~
\begin{minipage}{0.4\textwidth}
\begin{flushright} 
\large Bc. Štěpán \textsc{Bambula} \\ % Your name
\end{flushright}
\end{minipage}\\[1cm]

\end{titlepage}
% -------------------------------------------------------------------------------------------------------------


% =========================
% TITULNÍ STRANA 2222222222
% =========================

\begin{titlepage}
\center % Center everything on the page
 
%----------------------------------------------------------------------------------------
%	HEADING SECTIONS
%----------------------------------------------------------------------------------------

\textsc{\LARGE České vysoké učení technické v~Praze}\\[0.5cm] % Name of your university/college
\textsc{\Large Fakulta stavební}\\[0.5cm] % Major heading such as course name
\textsc{\large Studijní program: Geodézie a~kartografie}\\[0.5cm] % Minor heading such as course title
\textsc{\large Studijní obor: Geomatika}\\[0.5cm] % Minor heading such as course title

%----------------------------------------------------------------------------------------
%	LOGO SECTION
%----------------------------------------------------------------------------------------

\vspace{50pt}

\includegraphics[scale=1.2]{images/cvut_logo.png}  % Include a department/university logo 

%----------------------------------------------------------------------------------------
%	TITLE SECTION
%----------------------------------------------------------------------------------------

\vspace{40pt}

\textsc{\Large Diplomová práce}\\
\vfill
\textsc{\LARGE  Rozšíření nástroje pro práci s~katastrálními daty v~programu QGIS}\\[0.5cm] % Title of your document

\textsc{\Large QGIS VFK Plugin Improvements}
 
 
 %----------------------------------------------------------------------------------------
%	VEDOUCÍ PRÁCE
%----------------------------------------------------------------------------------------
 \vfill
\large Vedoucí práce: Ing. Martin \textsc{Landa}, Ph.D.

Katedra geomatiky
 
%----------------------------------------------------------------------------------------
%	AUTHOR SECTION
%----------------------------------------------------------------------------------------
\vfill
\begin{minipage}{0.4\textwidth}
\begin{flushleft} 
\large 2016	% datum
\end{flushleft}
\end{minipage}
~
\begin{minipage}{0.4\textwidth}
\begin{flushright} 
\large Bc. Štěpán \textsc{Bambula} \\ % Your name
\end{flushright}
\end{minipage}\\[1cm]
 
\end{titlepage}

%========================= ZADÁNÍ PRÁCE =================================
\clearpage
\pagestyle{empty}

\vspace*{\fill}
\begin{center}
\textsc{\Large Zde vložit zadání práce!!!}
\end{center}

\vspace*{\fill}

%========================= ABSTRAKT =====================================
\clearpage

\hfill

\noindent
\textsc{\Large Abstrakt}

\vspace{12pt}

Cílem diplomové práce je rozšířit projekt laboratoře OSGeoREL ČVUT v~Praze zaměřený na práci s~katastrálními daty poskytovanými ve výměnném formátu VFK v~prostředí open source nástroje QGIS. Práce navazuje na již existující nástroj implementovaný jako tzv. zásuvný modul a~rozšiřuje ho o~novou funkctionalitu a~to především zpracování a~vizualizaci datových vět změnových souborů VFK. Druhotným cílem je usnadnění distribuce zásuvného modulu v~prostředí QGIS s~důrazem na jeho přenositelnost.

\vspace{32pt}

\noindent
\textsc{\Large Klíčová slova}

\vspace{12pt}

VFK, QGIS, ČUZK, Python, C++, PyQt, GDAL, zásuvný modul


\vfill

\noindent
\textsc{\Large Abstract}

\vspace{12pt}


\vspace{32pt}

\noindent
\textsc{\Large Keywords}

\vspace{12pt}
VFK, QGIS, CUZK, Python, C++, PyQt, GDAL, plugin


\vfill

%========================= PROHLÁŠENÍ ==================================
\clearpage
\vspace*{\fill}

\noindent
\textsc{\Large Prohlášení}

\vspace{12pt}
Prohlašuji, že jsem diplomovou práci na téma \uv{Rozšíření nástroje pro práci s~katastrálními daty v~programu QGIS} vypracoval samostatně. Všechny podklady, ze kterých jsem čerpal, jsou uvedeny v~seznamu použité literatury.

\vspace{24pt}
\noindent
\begin{minipage}{0.4\textwidth}
\begin{flushleft}
\center 
V~Praze dne \dots \dots \dots \\
\end{flushleft}
\end{minipage}
~
\begin{minipage}{0.8\textwidth}
\begin{flushright} 
\vspace{20pt}
\center
\dots \dots \dots \dots \dots \dots \dots \dots \\
Štěpán Bambula
\end{flushright}
\end{minipage}\\[2cm]


%========================= PODĚKOVÁNÍ ==================================
\clearpage
\vspace*{\fill}

\noindent
\textsc{\Large Poděkování}

\vspace{12pt}

\vspace{2cm}

%======================POUZITE ZKRATKY===============================
\clearpage
%\rhead{SEZNAM POUŽITÝCH ZKRATEK}		 % vpravo název kapitoly
\chapter*{Seznam použitých zkratek}
\thispagestyle{empty}

\begin{description}
\item[VFK] Výměnný formát katastru nemovitostí
\item[ČUZK] Český úřad zeměměřický a~katastrální
\item[GDAL] Geospatial Data Abstraction Library
\item[GIS] Geografický informační systém
\item[OSGeo] Open Source Geospatial Foundation
\item[ISKN] Informační systém katastru nemovitostí
\end{description}


%=========================OBSAH=======================================
\clearpage
\rhead{{\rightmark}}		% vpravo název kapitoly
\tableofcontents
\thispagestyle{empty}

%============================ÚVOD====================================
\clearpage
\pagestyle{fancy}		% nastaví styl stránky a číslování
\setcounter{page}{1}   	% nastaví čítač stránek znovu od jedné
\pagenumbering{arabic} % číslování arabskými
\rhead{Úvod}		 % vpravo název kapitoly
\chapter*{Úvod}
\addcontentsline{toc}{chapter}{Úvod}




\clearpage
\rhead{{\rightmark}}
\chapter{Dostupné nástroje pro práci s~VFK}


\clearpage
\chapter{Použité technologie}

\section{QGIS}

QGIS je geografický informační systém, který je distribuován jako open-source\footnote{Open-source software je takový software, k~němuž zákazník dostane od jeho tvůrce zdrojový kód a~může jej dále upravovat. Jednotlivé definice termínu \uv{open source} se liší zvláště v~podmínkách pro další distribuci softwaru.\cite{abclinuxu_opensource}} pod licencí \textit{GNU General Public License}. Je oficiálním a~klíčovým produktem organizace OSGeo. Díky přenositelnosti zdrojového kódu je použitelný na širokém spektru platforem, ať už jsou to desktopové platformy Linux, MacOS, Windows, nebo mobilní platforma Android.

\begin{figure}[htb]
\centering
\includegraphics[scale=1]{images/qgis-logo.png}
\caption[QGIS -- logo]{QGIS -- logo (zdroj: \cite{qgis})}
\end{figure}

Program umožňuje prohlížení, tvorbu a~editaci velkého množství vektorových (Esri Shapefile, GeoJSON, GPX, \dots), ale i~rastrových (GeoTIFF, JPEG, \dots) nebo databázových formátů. Podporuje zpracování dat GPS a~tvorbu mapových výstupů. Mimo jiné umožňuje provádět prostorové analýzy, analýzy terénu nebo analýzy síťové, práci s~mapovou algebrou a~mnoho dalšího.

QGIS nedisponuje tak širokou paletou nástrojů, jako jeho open-source kolega GRASS GIS. Jeho funkcionalita ale může být rozšířena díky nepřebernému množství zásuvných modulů. Jedním z~nejdůležitějších modulů pro analýzu geografických dat je zásuvný modul GRASS GIS, který zpřístupňuje funkce stejnojmenného programu. QGIS poté může sloužit jako jeho nadstavba.
\cite{qgis}
\cite{qgis_wiki}


\section{GDAL/OGR}

GDAL je knihovna určená pro čtení a~zápis rastrových GIS formátů. Knihovna je vyvíjena pod hlavičkou Open Source Geospatial Foundation a~vydávána pod licencí \textit{X/MIT}. Knihovna používá jednoduchý abstraktní datový model pro všechny podporované datové formáty. Kromě toho nabízí také řadu užitečných nástrojů pro příkazovou řádku určených pro konverzi a~zpracování dat. \cite{gdal_wiki}

\begin{figure}[h]
\centering
\includegraphics[scale=1]{images/gdal-logo.png}
\caption[GDAL -- logo]{GDAL -- logo (zdroj: \cite{gdal})}
\end{figure}

GDAL byla původně vyvíjena Frankem Warmerdamem a~to do verze 1.3.2, posléze byla knihovna převedena na GDAL/OGR Project Management Committee, která je součástí Open Source Geospatial Foundation.\cite{gdal_wiki}

Knihovna OGR, která je od verze 2.0 součástí knihovny GDAL/OGR, slouží pro práci s~daty ve vektorovém formátu.\cite{gdal}

GDAL/OGR je považován za jeden z~hlavních open-source projektů. Knihovna je hojně využívána také v~komerční GIS sféře. Knihovna je otevřená a~poskytuje základní funkcionalitu potřebnou pro denní práci s~rozsáhlým množstvím GIS formátů.\cite{gdal_wiki}


\section{Python}

Jazyk Python je objektově orientovaný programovací jazyk, který efektivně používá víceúrovňové datové typy. Jedná se o~jazyk interpretovaný, čímž se jeví jako ideální nástroj pro psaní skriptů, ale i~rychlý vývoj aplikací. Je vyvíjen jako open-source software, díky čemuž se stává použitelným na velkém množství platforem (Linux, Windows, MacOS, \dots). Jazyk je rozšířitelný o~široké spektrum modulů, které umožňují řešit problematiku takřka z~jakékoli oblasti. V~současné době je Python vyvíjen ve dvou verzích, ve verzi 2.x a~v~novější verzi 3.x.
\cite{dive_into_python}
\cite{python_web}

\begin{figure}[htb]
\centering
\includegraphics[scale=1]{images/python-logo.png}
\caption[Python -- logo]{Python -- logo (zdroj: \cite{python_web})}
\end{figure}

\section{PyQt}

PyQt je modul, který zpřístupňuje knihovnu Qt pro programovací jazyk Python. Spolu s~PySide se jedná o~nejznámější a~nejpoužívanější modul pro Python postavený nad knihovnou Qt. Je vyvíjen britskou firmou Riverbank Computing ve dvou verzích. Ve verzi 4, podporující knihovnu Qt 4, a~ve verzi 5, která podporuje novější verzi Qt knihovny. Modul je dostupný na všech platformách, které podporují knihovnu Qt (Windows, MacOS/X a~Linux). PyQt je šířeno pod tzv. dvojí licencí, \textit{GNU GPL v3} a~\textit{Riverbank Commercial License}. Spolu s~těmito licencemi je dostupné i~pod komerční licencí.

\begin{figure}[htb]
\centering
\includegraphics[scale=1]{images/pyqt-logo.png}
\caption[PyQt -- logo]{PyQt -- logo (zdroj: \cite{pyqt_wiki})}
\end{figure}

Pro grafický návrh aplikace je vhodné použít nativní grafické uživatelské rozhraní Qt Designer. Výstupem z~tohoto programu je soubor obsahující vzhled aplikace ve formátu \textit{.xml}. PyQt je poté schopné tento formát převést do kódu jazyka Python. Pro komunikaci mezi objekty je využíváno signálů a~slotů, díky čemuž je vytvoření komponent velice snadné.

PyQt v~sobě kombinuje mocnost knihovny Qt s~jednoduchostí jazyka Python, což z~něj dělá výkonný nástroj pro vývoj grafických aplikací.
\cite{pyqt}
\cite{pyqt_wiki}



\clearpage
\chapter{Informační systém katastru nemovitostí}

ISKN je integrovaný informační systém pro podporu výkonu státní správy katastru nemovitostí a~pro zajištění jeho uživatelských služeb. Obsahuje prostředky pro současné vedení souborů popisných informací (SPI) a~souborů geodetických informací (SGI). Dále jsou v~něm obsaženy prostředky pro podporu správních a~administrativních činností při vedení katastru nemovitostí a~pro správu dokumentačních fondů. \cite{iskn}

\begin{figure}[htb]
\centering
\includegraphics[scale=1]{images/cuzk-logo.png}
\caption[ČUZK -- logo]{ČUZK -- logo (zdroj: \cite{iskn})}
\end{figure}

\section{Historie a~vývoj}

Vývoj systému byl započat v~roce 1997 ve spolupráci se společností APP Czech s.r.o.\footnote{Dnes společnost funguje pod názvem NESS Czech s.r.o.}, která fungovala jako systémový integrátor a~dodavatel aplikačního programového vybavení. Dalšími společnostmi podílejícími se na vývoji ISKN byly Infinity, a.s., Compaq Computer s.r.o.\footnote{Dnes pod názvem HP.}, Oracle Czech, s.r.o., Bentley Systems, s.r.o., BEA Systems, s.r.o. \cite{iskn}

Systém byl nasazen do provozu v~září roku 2001, a~to na všech katastrálních pracovištích včetně centrály. Dolaďování a~převzetí závěrečných etap probíhalo v~roce 2002. V~témže roce byl dokončen audit systému. \cite{iskn}

Implementace ISKN plně nahradila dřívější způsob vedení katastru nemovitostí. ISKN integroval vedení a~správu katastru nemovitostí pod jediný informační systém společný pro všechna pracoviště katastrálních úřadů a~centrum. Toto vede k~tomu, že je možné zveřejňovat a~poskytovat aktuální data z~katastru nemovitostí prostřednictvím dálkového přístupu během několika málo minut, a~to z~celého území republiky. \cite{iskn}

Data jsou do systému ISKN ukládána pomocí Spatial Cartridge Option do databáze Oracle. Podpora vzdáleného přístupu k~datům pomocí sítě Internet je zajištěna pomocí BEA WebLogic. Systémový management využívá nástrojů CA Unicenter. \cite{iskn}

V~roce 2004 byla uzavřena nová smlouva se společností NESS Czech s.r.o. na rozvoj a~údržbu informačního systému v~letech 2004 -- 2006. V~tomto období byl zmodernizován především Dálkový přistup do katastru nemovitostí a~zavedena orientační mapa parcel. Důležitou inovací  bylo zavedení elektronické značky pro výpis z~katastru nemovitostí a~pro kopii katastrální mapy \footnote{Tento krok umožnil, aby tzv. \uv{ověřující} podle zákona č. 365/2000 Sb., o~informačních systémech veřejné správy, v~platném znění, mohli poskytovat ověřené výpisy z~katastru nemovitostí, převedené z~elektronické do listinné podoby. \cite{iskn}}. \cite{iskn}

Společnost NESS Czech s.r.o. poté v~dalších letech vyhrála několik veřejných zakázek týkajících se údržby a~rozvoje ISKN. Hlavním cílem bylo převedení decentralizovaného systému (107 lokálních databází replikovaných do centrální databáze) na centralizovaný systém, ve kterém byla data ISKN uložena pouze v~jedné databázi. Spolu s~touto úpravou byla změněna i~architektura z~původní client/server na třívrstvou architekturu. Architektura je postavena na platformě Oracle Forms/Reports 10g a~databázi Oracle 10g. Další změnou byl přechod na vyšší verzi Bentley nástroje pro správu prostorových dat. \cite{iskn}

ISKN byl nadále zlepšován. Za zmínku stojí především systém pro Dálkový přístup do katastru nemovitostí nebo zavedení možnosti získat informaci o~ukončení řízení pomocí SMS nebo e-mailové zprávy. \cite{iskn}


\section{Hlavní charakteristiky ISKN}

\subsection{Optimalizace uložení dat}

Díky zvolení jednotného datového modelu pro uložení popisných a~prostorových dat v~databázi Oracle spolu s~daty týkajících se správních řízení byla umožněna současná aktualizace popisných a~prostorových dat a~udržení jejich vzájemné konzistence. Pro optimalizaci byla také přijata koncepce samostatné evidence budov a~bezešvé digitální katastrální mapy. Od konce roku 2001 jsou uchovávány také veškerá historická data popisných a~prostorových dat, díky čemuž je možné sestavovat data do potřebných výstupů k~historickému datu od zavedení ISKN v~roce 2001. \cite{iskn}

\subsection{Optimalizace procesů při správě KN}

Do systému ISKN byla zavedena celá řada automatických kontrol pro proces zapsání změny do KN. Dále bylo umožněno převzetí aktuálních dat z~jiných registrů (např. registr obyvatel) a~ostatních informačních systémů. Postup provedení změny dat KN je následující: na základě návrhu je připraven budoucí stav, který je možné před jeho zplatněním zobrazit (SPI, SGI), případně v~něm provádět úpravy. Toto zajišťuje důkladnou kontrolu výsledného stavu katastru. Proces realizace změny je navíc zajištěn i~technicko-organizačními opatřeními (návrh změny a~kontrolu, včetně zplatnění provádí vždy jiná osoba dle přidělených uživatelských rolí). \cite{iskn}

Díky novým procesům ve zpracování dat/návrhů změn je možné částečné nabytí platnosti geometrického plánu s~automatizovanou změnou návrhu změny v~budoucím stavu. Nové procesy také umožňují aktualizaci dat katastru nemovitostí takovým způsobem, aniž by zamkly aktualizovaná data. Pouze se jimi řeší konflikty v~aktualizaci stejných dat.

Součástí ISKN je také jednotná centrální správa číselníků, která vnáší jednotnost do procesu zpracování změn na katastrálních úřadech. Tímto se rapidně zvyšuje konzistence a~kvalita datové základny. Některé z~centrálních číselníků nebo seznamů jsou přebírány z~externích datových zdrojů (např. číselníky územní identifikace, PSČ). \cite{iskn}

\subsection{Bezpečnost}

Vysoká bezpečnost ochrany dat je zajištěna kombinací hardwarových prostředků s~operačním systémem, databází a~vlastní aplikací ISKN. Nepřetržitý provoz je zajištěn pomocí technologie databázových a~aplikačních clusterů a~tím, že je celá infrastruktura zdvojena (primární a~záložní centrum). Do záložního centra jsou replikována veškerá data tak, aby byl v~případě náhlého výpadku primárního centra zajištěn nepřetržitý provoz ISKN. \cite{iskn}


\section{Poskytování dat}

Poskytování dat je umožněno na základě vyhlášky číslo 358/2013 Sb., o~poskytování údajů z~katastru nemovitostí. \cite{iskn}

\subsection{Poskytování dat dálkovým přístupem}

Na základě registrace je umožněno poskytování dat (zdarma, nebo za úplatu podle typu zákazníka) prostřednictvím sítě Internet. Výpisy z~KN a~snímky katastrální mapy mají povahu veřejných elektronických listin (jsou opatřeny elektronickou značkou) a~mohou být převedeny do podoby listinných veřejných listin. Tímto způsobem je v~současné době vyřizována více než třetina výstupů. \cite{iskn}

Více informací o~této metodě poskytování dat je spolu s~aplikací dostupných na stránkách ČUZK (\url{http://www.cuzk.cz/aplikace-dp/}).

\subsection{Poskytování dat ve výměnném formátu ISKN}

Data z~KN mohou být poskytována v~textovém souboru, který obsahuje záznamy v~pevně definované struktuře. Více informací o~tomto výměnném formátu je uvedeno v~kapitole č. \ref{l_format_vfk}.


\clearpage
\chapter{Výměnný formát ISKN}
\label{l_format_vfk}

%TODO: http://freegis.fsv.cvut.cz/gwiki/V%C3%BDm%C4%9Bnn%C3%BD_form%C3%A1t_ISKN
%TODO: http://geo.fsv.cvut.cz/~landa/publications/2005/diploma_thesis/martin.landa-thesis.pdf
%TODO: http://www.cuzk.cz/Katastr-nemovitosti/Poskytovani-udaju-z-KN/Vymenny-format-KN/Vymenny-format-ISKN-v-textovem-tvaru.aspx
%TODO: http://www.cuzk.cz/Katastr-nemovitosti/Poskytovani-udaju-z-KN/Vymenny-format-KN/Vymenny-format-ISKN-v-textovem-tvaru/Popis_VF_ISKN-v5_1-1-(1).aspx





\cite{dp_landa}
\cite{vfk_struktura}












%======================CITACE=========================================
\clearpage
\rhead{{\rightmark}}		% vpravo název kapitoly
\renewcommand{\refname}{Použitá literatura}
\bibliography{citace}
\bibliographystyle{czechiso}

%======================SEZNAM OBRÁZKŮ===============================
\clearpage
\listoffigures

%======================SEZNAM TABULEK================================
\clearpage
\listoftables

%====================== PŘÍLOHY ========================================
\newpage
\appendix

\setcounter{page}{1}   	% nastaví čítač stránek znovu od jedné
\pagenumbering{Roman} % číslování arabskými

%-----------------------------------------------


\end{document}